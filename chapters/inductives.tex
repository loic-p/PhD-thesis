\setchapterimage[6cm]{seaside}
\setchapterpreamble[u]{\margintoc}
\chapter{Extensions of \SetoidCC}
\labch{extensions}

In this chapter, we investigate several variants and extensions of \SetoidCC.

In \cref{sec:cast-refl}, we add a computation rule to the \( \castName \)
operator so that it reduces when applied to a reflexive identity proof. 
We propose a normalization algorithm for the resulting system, and we 
conjecture that it always terminates on well-typed terms.

In \cref{sec:heterogenous}

Finally, in \cref{sec:proof-rel-impred} we add a universe \( \Prop \) of 
proof-\emph{relevant} impredicative types to \SetoidCC to increase
its computational power.
% 
We show that if we extend the observational equality and the \( \castName \)
operator to this universe, type-checking becomes undecidable.

\section{Cast and Reflexivity}
\label{sec:cast-refl}

\subsection{Observational versus Inductive}

In \SetoidCC, we have a bit of an awkward trade-off between two different 
notions of propositional equality.

On the one hand, we have the observational equality \( \Obseq[A]{t}{u} \),
a proof-irrelevant proposition.
% 
The observational equality has strong selling points: not only does it validate 
the principle of Uniqueness of Identity Proofs (UIP) by definition because of 
computational irrelevance, but it also is equipped with reduction rules that 
bake in useful reasoning principles such as function extensionality and proposition 
extensionality.

But being computationally irrelevant also has its drawbacks. Because the equality
proofs pack no information, the eliminator \( \castName \) computes on types constructors, 
which means that it will fully reduce only when applied to closed types.
% 
To make up for this weakness, \SetoidCC provides the axiom \( \castreflName \) 
which is a proof of observational equality between a reflexive cast and the
identity.
% 
\sideremark{Note that the rule for \( \castreflName \) is phrased with an 
  arbitrary proof of equality between \( A \) and itself. This is equivalent
  to the statement using \( \metaop{refl} \) because of proof irrelevance.}
% 
\begin{mathpar}
	\inferrule[Cast-Refl]
		{%\tytm{\Gamma}{A}{\Univ_i}
			\tytmannotated{\Gamma}{t}{A}{\Type_i} \\
			\tytmannotated{\Gamma}{e}{\Obseq[{\Type_i}]{A}{A}}{\sProp}}
		{\tytmannotated{\Gamma}{\castrefl{A}{t}}{\Obseq[A]{t}{\cast{A}{A}{e}{t}}}{\sProp}}
\end{mathpar}

On the other hand, we have the inductive equality \( \Id{A}{t}{u} \), which is
a proof-relevant analogue of the Martin-Löf identity type --
% 
and as such, it comes equipped with the usual \( \metaop{J} \) eliminator and its
computation rule. 
% 
\begin{mathpar}
  \inferrule{\tytm{\Gamma}{A}{\Type_i}
			\\ \tytm{\Gamma}{t}{A}
			\\ \tytm{\Gamma}{B}{\Depfun{A}{\Fun{\Id{A}{t}{x}}{\Type_j}}}
			\\ \tytm{\Gamma}{u}{B\ t\ \id{t}}}
			{\red{\Gamma}{\J{A}{t}{B}{u}{t}{\id{t}}}{u}{B\ t\ \id{t}}}
\end{mathpar}

But it is also better than the Martin-Löf identity type, as it validates the 
signature extensionality principles of \SetoidCC, which are namely UIP, function 
extensionality and proposition extensionality.
% 
However, computing on reflexivity comes at a price. For the inductive equality,
the principle of UIP only holds up to a propositional equality, and the same 
goes for function extensionality and proposition extensionality.

All in all, the difference is rather subtle. The two relations are logically 
equivalent, but they do not satisfy the exact same computation rules.
% 
As if the distinction between propositional and definitional equality 
wasn't confusing enough for newcomers to type theory!

\subsection{The System \SetoidCCplus}

In hope to reconciliate these two propositional equalities and get the best of 
both worlds, we will attempt to extend \SetoidCC with a \( J \) eliminator for 
the observational equality that supports the computation rule. 

First, we remark that the problem can be reduced to modifying the behavior of 
the \( \castName \) operator, so that it validates 
rule~\nameref{infrule:cast-refl-plus}.
% 
\begin{mathpar}
	\inferrule[Cast-Refl+]
		{%\tytm{\Gamma}{A}{\Univ_i}
			\tytmannotated{\Gamma}{t}{A}{\Type_i} \\
			\tytmannotated{\Gamma}{e}{\Obseq[{\Type_i}]{A}{A}}{\sProp}}
		{\eqtmannotated{\Gamma}{\cast{A}{A}{e}{t}}{t}{A}{\Type_i}}
  \ilabel{infrule:cast-refl-plus}
\end{mathpar}

Indeed, the more general \( J \) eliminator can be defined from 
\( \castName \) as described in \cref{sec:inductive-equality}, and in presence
of rule~\nameref{infrule:cast-refl-plus} it will satisfy the computation rule 
on reflexivity as a definitional equality.

We use the name \SetoidCCplus for the variant of the observational calculus of
constructions that supports rule~\nameref{infrule:cast-refl-plus}.
% 
Naturally, this extends the convertibility relation, meaning that we
need new algorithms to decide conversion for \SetoidCCplus terms.
% 
We will describe two such algorithms below, but unfortunately we are not
able to supplement either with a proof of correction. 
We leave these as conjectures for further work.

\paragraph{Implementing \nameref{infrule:cast-refl-plus} with reduction}

The first idea that comes to mind is to add new reduction rules for 
\( \cast{A}{B}{e}{t} \), so that it will actually reduce to \( t \) in case the
types \( A \) and \( B \) are convertible.
\begin{mathpar}
	\inferrule[Cast-Refl-Neutral]
		{%\tytm{\Gamma}{A}{\Univ_i}
			\tytmannotated{\Gamma}{t}{A}{\Type_i} \\
      \tytm{\Gamma}{B}{\Type_i} \\
			\tytmannotated{\Gamma}{e}{\Obseq[{\Type_i}]{A}{B}}{\sProp} \\
      \eqtm{\Gamma}{A}{B}{\Type_i} \\
      \text{\( A \) and \( B \) neutral}}
		{\red{\Gamma}{\cast{A}{B}{e}{t}}{t}{A}}
  \ilabel{infrule:cast-refl-neutral}
\end{mathpar}

A consequence is that now reduction is intertwined with conversion checking,
which makes for meta-theory that is much more difficult.
 
Type casting operators that reduce on identical types have a long history in
type theory. 
% 
In the context of impredicative systems, they were notably studied by Girard 
for System F \sidecite{girard72}, and by Werner for \CIC \sidecite{Werner2008}.
% 
In both cases, it unfortunately turns out that the resulting system loses 
normalization for open terms \sidecite{lmcs:6606}.

This incompatibility between impredicativity and type casting can be 
interpreted as a consequence of the breakage of parametricity.
%
Indeed, the operator \( {\castName : (A\ B\ : \Type_i) \to A \to B} \) 
is fundamentally anti-parametric, because in case \( A \equiv B \) it
reduces to \( t \) which satisfies the parametricity predicate associated 
to \( A \) but not necessarily the predicate associated to \( B \).
% 
But normalization models for impredicative type theories are usually built
using reducibility candidates, which rely crucially on parametricity to 
work.

Fortunately, we do not expect that the same problems apply to our reduction 
rules for \SetoidCCplus, because \( \castName \) only computes in the 
predicative fragment and proof-irrelevance once again comes to our rescue 
in the impredicative layer.
% 
Therefore, we feel justified in conjecturing that this system is in fact 
normalizing.
% 
\begin{conjecture}
  \SetoidCCplus with computation rules for \( \castName \) on neutral types is 
  normalizing.
\end{conjecture}

As further evidence for this conjecture, we remark that the situation is very 
similar to reduction rules for \( \castName \) on type constructors 
(\ie the rules of \SetoidCC) which are certainly not parametric either.
% 
In \cref{sec:proof-rel-impred}, we will show that extending these rules to a
proof-relevant impredicative universe breaks normalization, but we were still
able to prove normalization for \SetoidCC in \cref{ch:metatheory}. 

\paragraph{Implementing \nameref{infrule:cast-refl-plus} in conversion only}



\section{Heterogeneous Equality}

\section{General Inductive Schemes}

\subsection{Parameters and Universes}

\section{Proof-Relevant Impredicativity}
\label{sec:proof-rel-impred}

In dependent type theory, a sort \( \sProp \) is said to be
\emph{impredicative} if it is closed under dependent products over any index
type: for all types \( A \) and functions \( \tm{B}{A \to \sProp} \),
the dependent product \( \Depfun{A}{B\ x} \) is in \( \sProp \).
%
In particular, impredicativity allows the definition of self-referential propositions,
which quantify over the type \( \sProp \) of all propositions and may thus
be applied to themselves.
%
In addition to providing a tremendous amount of logical power, impredicativity is
a crucial ingredient in common mathematical constructions, such as
Tarski's fixed point theorem or lattice theory~\sidecite{paco}.
%
On the other side of the coin, a predicative hierarchy $(\varType_i)_{i \in \Nat}$
requires dependent products to inhabit a higher universe level
than their domain and codomain: for all types \( A : \varType_i \) and functions
\( \tm{B}{A \to \varType_j} \), the dependent product
\( \Depfun{A}{B\ x} \) is in \( \varType_{max(i,j)} \).
%
While the latter is easier to model, as the universes $\varType_i$ can be
constructed incrementally by induction on the level $i$, it results
in a theory that is less flexible in practice, whether the mention of
levels is explicit (as in \Agda) or implicit (as in \Coq).
%
Impredicative propositions make for an altogether
more comfortable framework, in which less universe levels have to
be dealt with.

While impredicativity is a prominent feature of the Calculus of
Constructions~\sidecite{Coquand-CC}, and by extension of \Coq, it is
absent from the standard presentation of Martin-Löf type theory
(\MLTT~\sidecite{MARTINLOF197573}), and not available in \Agda.
%
This reluctance may be explained by the sheer difficulty of designing models
to reason about impredicative theories, and by the numerous incompatibility
results, from \sidecitet{hurkens95} paradox to the more recent proof by
\sidecitet{lmcs:6606} that a naive implementation of uniqueness of identity
proofs (UIP) via definitionally proof-irrelevant equalities breaks \Coq's
normalization algorithm on impredicative propositions.

However, as noted by Abel and Coquand, it is not clear whether that last
result stems from a deep incompatibility between UIP and impredicativity,
or if it is an artifact of an inadequate type-checking algorithm.
%
Clarifying this issue seems especially important given the renewed
interest in definitional proof-irrelevance and UIP shown by the community
in recent years~\sidecite{sterling_et_al:LIPIcs:2019:10538,pujet:hal-03367052,Altenkirch2019}.
%
For instance, the recent presentation \SetoidTT of observational
equality of \sidecitet{pujet:hal-03367052} has an equality which can be
eliminated using a cast operator that computes differently from the
usual eliminator of Martin-Löf identity type, as used by \sidecitet{lmcs:6606}.


In this paper, we show that impredicativity is harmless when confined
to a sort of definitionally proof-irrelevant propositions, as it is never
necessary to compute with irrelevant proof terms.
%
As a result, we are able to extend \SetoidTT with impredicativity while
preserving definitional UIP and all extensionality principles that come
with an observational type theory (such as propositional or function
extensionality).
%
The resulting system, dubbed \SetoidCC, still enjoys
normalization via inductive-recursive logical relations---no
realizability model is needed---and decidability of type-checking.
%
This result has been formalized in \Agda using the framework of
\sidecitet{Abel:POPL2018}.\footnote{see anonymous supplementary
  materials.}
%
The fact that a proof scheme that has been designed with predicative
theories in mind carries through in an impredicative setting
may come as a surprise.
%
The crux of the proof is to realize that we can get away with
remarkably little structure in the interpretation of the impredicative
dependent products and sums, to the extent that we do not even mention the
interpretation of the domain and codomain.
%
We manage to prove the soundness of our normalization model
despite this weaker induction hypothesis by using a \emph{paranoid}
version of the type system, which can then be proven equivalent to the
\emph{economic} version (a terminology introduced by
\sidecitet{andrej17:paranoid}).

%
Furthermore, we prove that normalization for \SetoidCC can be
carried in plain Martin-Löf type theory with indexed inductive types,
thereby showing that this impredicative universe does not contribute
to the computational power of the proof-relevant fragment at all,
despite the increase in logical power.
%
We also investigate the possibility of recovering the lost computational power
from a proof-relevant impredicative universe, hoping that the different
computational behaviour of \SetoidCC would allow us to circumvent Abel and
Coquand's argument, but alas, it does not. In fact, a slightly modified
argument shows undecidability of type-checking in presence of a
proof-relevant impredicative universe.
%
Therefore, we exhibit a tradeoff between the additional comfort of
UIP and extensionality principles, or the possibility to compute with
impredicative functions.

Finally, because our normalization proof does not imply logical
consistency, we define a model of \SetoidCC in ZF Set
theory.
%
This shows that the computational content of \SetoidCC can be studied
without impredicativity in the metatheory, while the study of the
logical content of \SetoidCC fundamentally requires impredicativity in
the metatheory.

\paragraph{Plan of the paper}

In \cref{sec:an-impr-type}, we present the syntax and typing rules of
\SetoidCC, focusing on the differences with respect to \SetoidTT.
%
Then, we develop the logical relation
framework that shows normalization and decidability of conversion for
\SetoidCC (\cref{sec:logic-relat-with}).
%
In \cref{sec:analys-norm-proof}, we show that \SetoidCC and \MLTT with
indexed inductive types can
actually express the exact same integer functions
  as closed terms of type \( \Fun{\Nat}{\Nat} \) by simplifying the
  logical relation framework to avoid the use of induction-recursion.
  %
  Finally, we show logical consistency of \SetoidCC in
  \cref{sec:cons-seto-model} and adapt the undecidability argument in
  presence of a proof-relevant impredicative universe of
  \sidecitet{lmcs:6606} to our setting in \cref{sec:setoidcc-with-an}.

  \section{\SetoidCC with an imprecative universe has undecidable type-checking}
  \label{sec:setoidcc-with-an}
  
  The system we have presented so far has an impredicative universe
  \( \sProp \) of \emph{strict} propositions, which contains the observational
  equality and the false proposition with the corresponding large
  elimination principles.
  %
  In other words, \( \sProp \) provides the user with a complete Heyting
  algebra of truth values, as is frequently assumed even in constructive
  mathematics.
  % TO BE DONE \sidecite{TODO}.
  
  However, \( \sProp \) is still a bit weaker than the impredicative
  universe \( \Prop \) of the Calculus of Inductive Constructions:
  \( \Prop \) allows large elimination for all the \emph{singleton}
  inductive types, that is the inductive types with only one constructor
  whose arguments all have sort \( \Prop \).
  %
  This includes equality, but also the accessibility predicate from which
  we can obtain some constructive choice principles (as discussed in \sidecite{forster:LIPIcs:2021:13455} that may
  come in handy for mathematics.
  
  Thus one may wonder if we can add \( \Prop \) to \SetoidCC alongside the
  universe of strict propositions, using the same computation rules as for the
  proof-relevant fragment of \SetoidCC so that it can provide us with
  more logical power while still enjoying extensionality principles.
  %
  We need to be careful though, as past experiments of adding a
  proof-irrelevant equality to Coq that have been mentioned
  in~\sidecite{gilbert:hal-01859964} have resulted in breakage of
  normalization of open terms~\sidecite{lmcs:6606}.
  %
  
  However the argument of Abel and Coquand relies on the fact that
  equality behaves as the usual Martin-Löf Identity Type, which is an
  inductive type that computes via the \( J \)-eliminator.
  %
  In \SetoidCC, elimination of equality is done using the cast operator
  which is quite different, as it computes by comparing the normal forms
  of the source and target types, which intertwines normalization and
  conversion checking in sophisticated ways.
  %
  Unfortunately, it turns out that \SetoidCC is not compatible with \( \Prop \)
  either.
  %
  In fact, we show in this section that it is possible to use a modified
  version of Abel and Coquand's argument to show undecidability of
  type-checking for open terms in presence of \( \Prop \), both for
  \SetoidCC and the theory of \sidecitet{lmcs:6606}.
  
  \subsection{Outline}
  
  The idea of the proof is rather simple:
  %
  on the one hand, impredicativity makes it possible to give a type to the term
  \[
  \Delta_f := \lambda x .\ f\ (x\ x)
  \]
  for any well-typed function \( f \) of type \( \Fun{(\Fun{\Nat}{\Nat})}{(\Fun{\Nat}{\Nat})} \), which is one half of the fixed point
  combinator \( Y_f := \Delta_f\ \Delta_f \).
  %
  On the other hand, \( \castName \) can be used to convert between any two types
  in an inconsistent context, in a way that does not block reduction.
  This allows us to apply \( \Delta_f \) to itself (up to a cast) and obtain
  an approximation of \( Y_f \).
  %
  Then, using these pseudo-fixed points, we can build a term that loops through the
  iterates of any integer function:
  %
  \begin{theorem}
    Let \( g \) be any closed term of type \( \Fun{\Nat}{\Nat} \), where \( \Nat \)
    denotes the inductive natural numbers in \( \Prop \).
    %
    In an inconsistent context, one can define a term \( \mathrm{dec}_g \)
    of type \( \Nat \) which is convertible to 0 if and only if there is a positive integer
    \( n \) such that \( g^n\ 1 = 0 \), and diverges otherwise.
  \end{theorem}
  %
  Of course, the existence of such an integer is not decidable in general,
  thus conversion and typing are undecidable for \SetoidCC + \( \Prop \).
  %
  As an aside, note that this construction can also be carried out in
  \( \sProp \) if we add proof-irrelevant natural numbers. But it does
  not cause any issue, as we have a simple way to check for convertibility of
  proof-irrelevant terms: having the same type is sufficient! For this reason,
  we never perform reduction in the proof-irrelevant layers, and need not
  to worry about these potentially divergent terms.
  
  \subsection{Definition of an Approximate Fixed Point Combinator}
  
  Let \( f \) be a closed function of type \( \Fun{(\Fun{\Nat}{\Nat})}{(\Fun{\Nat}{\Nat})} \).
  %
  In an inconsistent context, we can derive a term
  \[
     e : \Depfun[X\ Y]{\Prop}{\Obseq{X}{Y}}
  \]
  that allows us to freely cast between any two types.
  %
  We can use it to derive our approximate fixpoint combinator by defining
  \begin{align*}
    \bot &:= \Depfun[X]{\Prop}{X}\\
    \Delta_f &:= \lam{\bot}{f\ (x\ (\Fun{\bot}{\Fun{\Nat}{\Nat}})\ x)}\\
    \Delta_f' &:= \lam[X]{\Prop}{\cast{\Fun{\bot}{\Fun{\Nat}{\Nat}}}{X}{e}{\Delta_f}}\\
    Y_f &:= \Delta_f\ \Delta_f'
  \end{align*}
  Technically, it turns out that \( Y_f \) does not exactly behave like
  a fixpoint combinator because of additional casts appearing during the
  reduction.
  %
  A tedious but straightforward computation in \SetoidCC shows that
  \begin{align*}
    Y_f &= \Delta_f\ \Delta_f'\\
        &= f\ (\alpha\ (\Delta_f\ (\beta\ \Delta_f')))\\
        &= f\ (\alpha\ (f\ (\alpha^2\ (\Delta_f\ (\beta^3\ \Delta_f')))))\\
        &= f\ (\alpha\ (f\ (\alpha^2\ (f\ (\alpha^4\ (\Delta_f\ (\beta^7\ \Delta_f')))))))\\
        &= ...
  \end{align*}
  where \( \alpha := \lambda x.\ {\cast{\Fun{\Nat}{\Nat}}{\Fun{\Nat}{\Nat}}{e}{x}} \)
  and \( \beta := \lambda x.\ {\cast{\bot}{\bot}{e}{x}} \).
  
  However, this behavior is sufficient for our purposes, since all the \( \alpha \) disappear
  when the function is applied to an actual integer.
  %
  Therefore, given a closed function \( g \) of type \( \Fun{\Nat}{\Nat} \) that
  terminates on all integer inputs, we instantiate \( f \) with
  \[
    \lam[p]{\Fun{\Nat}{\Nat}}{\lam[n]{\Nat}{{\natrec{\Nat}{0}{p\ (g\ n)}{n}}}}
  \]
  and we obtain that \( Y_f\ 1 \) reduces to 0 if there exists a \( n \) such that
  \( g^n\ 1 = 0 \), and diverges (for any reduction strategy) otherwise.
  
  
  By defining $\castName$ using the elimination principle of the
  inductive equality, the same result applies to the
  theory of \sidecitet{lmcs:6606}, even if the reduction of $Y_f$ is
  slightly different.
  
  \subsection{Conversion is Undecidable}
  
  \SetoidCC is expressive enough to encode Turing machines so that
  knowing whether there exists a \( n \) such that \( g^n\ 1 = 0 \) is
  undecidable in general.
  %
  To complete the proof of undecidability of conversion, we still need
  to show that a term that diverges for any reduction strategy is not
  convertible to $0$. In other words, we need to show that conversion is
  not degenerate.
  %
  To establish this, we consider a variation on the standard results of untyped
  lambda-calculus (\sidecite{Barendregt}) showing that diverging terms
  cannot be equal to a normal form, by using confluence of the system
  and the fact that equality of two terms $t$ and $u$ corresponds to the
  existence of a zig-zag of reduction $t \Rightarrow^* t_1 \Leftarrow^*
  t_2\Rightarrow^*\dots \Leftarrow^* u$.
  
  In \CIC, confluence of reduction has been formally proven by
  \sidecitet{coqcoqcorrect}, by using the standard Tait-Martin Löf
  method~\sidecite{takahashi1989parallel} making use of parallel
  reduction.
  %
  Extending the proof of confluence for \SetoidCC is straightforward for
  the reduction involving the cast operator and the equality type, as
  they do not add any critical pair.
  
  The only difficulty is to additionally account for $\eta$-conversion
  and proof-irrelevance in the conversion to get a similar notion of
  zig-zag.
  %
  Actually, $\eta$-conversion can be turned into an
  $\eta$-reduction rule and for proof-irrelevance, we can use the fact
  that because no reduction depends on the shape of a proof-irrelevant
  terms, proof-irrelevance commutes with reduction.
  So in \SetoidCC + $\Prop$, any proof of conversion between $t$ and $u$
  can be seen as a zig-zag up to proof-irrelevance
  %
  $t \Rightarrow^* t_1 \equiv_{pi} t_1' \Leftarrow^*
  t_2\Rightarrow ^* t_2' \equiv_{pi} \dots \Leftarrow^* u$ where $\equiv_{pi}$ is syntactic
  equality up-to applications of proof-irrelevance.
  %
  From this, using confluence, we conclude to any term that it convertible to $0$ can be
  reduced to $0$, so a term that diverges for any reduction strategy
  cannot be convertible to $0$.
  
  Again, a similar proof can be done in the theory of \sidecitet{lmcs:6606}.