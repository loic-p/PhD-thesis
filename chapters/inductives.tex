\setchapterimage[6cm]{seaside}
\setchapterpreamble[u]{\margintoc}
\chapter{Extensions of \SetoidCC}
\labch{layout}

\shepherd{Inductive scheme, Heterogeneous equality, Cast-Refl definitional, Impredicative impossibility}

So far, we have defined and studied a minimal version of \SetoidTT.
%
In this section, we consider several extensions of the theory.
%
In regular \MLTT, adding a type generally means giving rules for type
formation, introduction, elimination and computation of the
eliminator. In our case, we also need to provide computation rules for
equality of types, equality of terms, and cast. These extensions have
not been formalized in the companion \Agda development.

\subsection{Quotient Types}

Quotients are a ubiquitous construction in mathematics, and one that
is famously difficult to handle smoothly in \MLTT.
%
The usual way to handle quotients is \textit{via} setoids, but since
this structure is not built in \MLTT, all the functions between
setoids, all the predicates, etc... have to be supplemented with
equality preservation lemmas---which appears to be quickly unmanageable.

In \SetoidTT however, every type is already a setoid and every term
preserves the setoid equality by construction. This is a very
comfortable setting for quotients, that can thus be added to the
theory---provided the relation that induces the quotient is
proof-irrelevant.
%
This can be seen as a limitation, as noticed
by~\sidecitet{sterling_et_al:LIPIcs:2019:10538}, as it is generally
impossible to extract proof-relevant information from equality in
the quotient type. This is in contrast with the development of higher
inductive types in the cubical setting~\sidecite{10.1145/3209108.3209197}.
%
On the other hand, the positive consequence of this limitation is
that the elimination principle of the quotient types is fairly easy to
manipulate in \SetoidTT.

Quotient types are defined on a type \( A \) equipped with an
equivalence relation on \( A \)
{\small
  \[
  \inferrule{\tytm{\Gamma}{A}{\varType_i}
            \\ \tytm{\Gamma}{R}{\Fun{A}{\Fun{A}{\sProp}}}
            \\ \tytm{\Gamma}{R_r}{\Depfun{A}{R\ x\ x}}
            \\ \tytm{\Gamma}{R_s}{\Depfun[x,y]{A}{\Fun{R\ x\ y}{R\ y\ x}}}
            \\ \tytm{\Gamma}{R_t}{\Depfun[x,y,z]{A}{\Fun{R\ x\ y}{\Fun{R\ y\ z}{R\ x\ z}}}}}
            {\tytm{\Gamma}{\Quo{A}{(R,R_r,R_s,R_t)}}{\varType_i}}
          \]
          }
Since the proofs of reflexivity, symmetry and transitivity appear
everywhere but are proof-irrelevant, we will generally omit them in the
assumptions of the rules, and write \( \Quo{A}{R} \) instead of
\( \Quo{A}{(R,R_r,R_s,R_t)} \).
%
The only constructor of quotient types is the canonical projection: from an element \( t \) of
\( A \), one obtains an element \( \quo{t} \) of \( \Quo{A}{R} \) that
is whnf, and equality between
two canonical projections reduces to \( R \).
%
{\small
  \begin{mathpar}
  \inferrule{\tytm{\Gamma}{t}{A}}
            {\tytm{\Gamma}{\quo{t}}{\Quo{A}{R}}}
  \and
  \inferrule{\tytm{\Gamma}{t}{A}
            \\ \tytm{\Gamma}{u}{A}}
            {\red{\Gamma}{\Obseq[\Quo{A}{R}]{\quo{t}}{\quo{u}}}{R\ t\ u}{\sProp[i]}}
          \end{mathpar}
}
The definition of $\castName$ between two quotient types reduces when
the casted term is a canonical projection.
%
{\small
\begin{mathpar}
  \inferrule{\tytm{\Gamma}{e}{\Obseq[\varType]{\Quo{A}{R}}{\Quo{A'}{R'}}}
    \\ \tytm{\Gamma}{t}{A}}
  {\red{\Gamma}{\cast{\Quo{A}{R}}{\Quo{A'}{R'}}{e}{\quo{t}}}{\quo{\cast{A}{A'}{\fst{e}}{t}}}{\Quo{A}{R'}}}
\end{mathpar}
}

Observational equality between two quotient types reduces to equality
of the (proof-relevant part of the) telescopes that define each
quotient:
{\small
\[
  \inferrule{\tytm{\Gamma}{A}{\varType_i}
    \\ \tytm{\Gamma}{R}{\Fun{A}{\Fun{A}{\sProp}}}
    \\ \tytm{\Gamma}{A'}{\varType_i}
    \\ \tytm{\Gamma}{R'}{\Fun{A'}{\Fun{A'}{\sProp}}}}
    {\red{\Gamma}{\Obseq[\varType]{\Quo{A}{R}}{\Quo{A'}{R'}}}
      {\Exists[e]{\Obseq[\varType]{A}{A'}}
      {\Depfun[x\ y]{A}
      {\Obseq[\sProp]{R\ x\ y}{R\ \cast{A}{A'}{e}{x}\ \cast{A}{A'}{e}{y}}}}}{\sProp[i]}}
\]
}
The eliminator for quotient types encodes the universal property of quotients: to construct a
function out of a quotient \( \Quo{A}{R} \), it suffices to give a function \( t_\pi \) out of \( A \)
such that if \( R \ x \ y \), then their images under \( t_\pi \) are equal.
{\small
\[
  \inferrule{\tytm{\Gamma}{B}{\Fun{\Quo{A}{R}}{\Type[i]{}}}
            \\ \tytm{\Gamma}{t_\pi}{\Depfun{A}{B\ \quo{x}}}
            \\ \tytm{\Gamma}{t_\sim}{\Depfun[x,y]{A}{\Depfun[e]{R\ x\ y}{\Obseq[B\ \quo{x}]{(t_\pi\ x)}{\cast{B\ \quo{y}}{B\ \quo{x}}{B\ \sym{e}}{t_\pi\ y}}}}}
            \\ \tytm{\Gamma}{u}{\Quo{A}{R}}}
            {\tytm{\Gamma}{\quorec{B}{t_\pi}{t_\sim}{u}}{B\ u}}
          \]}
The eliminator for quotient types has the obvious computation rule
{\small
\[
  \inferrule{\tytm{\Gamma}{B}{\Fun{\Quo{A}{R}}{\Type[i]{}}}
            \\ \tytm{\Gamma}{t_\pi}{\Depfun{A}{B\ \quo{x}}}
            \\ \tytm{\Gamma}{t_\sim}{\Depfun[x,y]{A}{\Depfun[e]{R\ x\ y}{\Obseq[B\ \quo{x}]{(t_\pi\ x)}{\cast{B\ \quo{y}}{B\ \quo{x}}{B\ \sym{e}}{t_\pi\ y}}}}}
            \\ \tytm{\Gamma}{u}{A}}
            {\red{\Gamma}{\quorec{B}{t_\pi}{t_\sim}{\quo{u}}}{t_\pi\ u}{B\ (\quo{u})}}
\]}
%
There are also reduction rules that reduce terms to a weak head normal
form under equality (of the form $\Obseq[\varType]{\Quo{A}{R}}{X}$, and
$\Obseq[\Quo{A}{R}]{x}{y}$, $\Obseq[\Quo{A}{R}]{\quo{a}}{y}$), and
similarly for cast and quotient elimination,
as well as new congruence rules.
% ifAppendix{ as defined in \cref{fig:SetoidTT-substitution,fig:SetoidTT-congruence}}

We now turn to the proof that the metatheoretical properties of
\SetoidTT are preserved by the addition of quotient types. More
explicitly, we extend the logical relation framework and its fundamental lemma, as well as the setoid model,
to quotient types.
%
From there, we can replay our proofs of consistency, normalization,
canonicity and decidability.

\paragraph{Reducibility}

We first extend the reducibility proof to quotient types. Since they add a new family of
types in normal form, we need to add a case to the logical relation:
{\small
\[
  \inferrule{\redT{\Gamma}{A}{\Quo{A'}{(R,R_r,R_s,R_t)}}{\varType_i}
            \\ \tytm{\Gamma}{A'}{\varType_i}
            \\ \forall \wkrho.\ \ \tytyR{\ell}{\Delta}{A'[\rho]}
            \\ \tytmR{\ell}{\Gamma}{R}{\Fun{A'}{\Fun{A'}{\sProp}}}
            \\ \tytm{\Gamma}{R_r}{\Depfun{A'}{R\ x\ x}}
            \\ \tytm{\Gamma}{R_s}{\Depfun[x,y]{A'}{\Fun{R\ x\ y}{R\ y\ x}}}
            \\ \tytm{\Gamma}{R_t}{\Depfun[x,y,z]{A'}{\Fun{R\ x\ y}{\Fun{R\ y\ z}{R\ x\ z}}}}}
            {\tytyR{\ell}{\Gamma}{A}}
\]}
Given a type $A$ reducible to a quotient type, we
define:
\begin{itemize}
  \item \( \eqtyR{\ell}{\Gamma}{A}{B} \) if there are terms \( B', Q, Q_r, Q_s, Q_t \) such that
    \begin{itemize}
      \item \( \redT{\Gamma}{B}{\Quo{B'}{(Q,Q_r,Q_s,Q_t)}}{\varType_i} \)
      \item \( \forall \wkrho.\ \ \eqtyR{\ell}{\Delta}{A'[\rho]}{B'[\rho]} \)
      \item \( \eqtmR{\ell}{\Gamma}{R}{Q}{\Fun{A'}{\Fun{A'}{\sProp}}} \).
      % \item \( \tytm{\Gamma}{Q_r}{\Depfun{B'}{Q\ x\ x}} \)
      % \item \( \tytm{\Gamma}{Q_s}{\Depfun[x,y]{B'}{\Fun{Q\ x\ y}{Q\ y\ x}}} \)
      % \item \( \tytm{\Gamma}{Q_t}{\Depfun[x,y,z]{B'}{\Fun{Q\ x\ y}{\Fun{Q\ y\ z}{Q\ x\ z}}}}} \)
    \end{itemize}
  \item \( \tytmR{\ell}{\Gamma}{t}{A} \) if there is a normal form \( t' \) such that
    \( \redT{\Gamma}{t}{t'}{\Quo{A'}{R}} \) and \( \quotmR{\Gamma}{t'} \), which is defined by
{\small
    \[
      \inferrule{\tytmR{\ell}{\Gamma}{t}{A'}}
                {\quotmR{\Gamma}{\quo{t}}}
      \quad
      \inferrule{\tytm{\Gamma}{n}{\Quo{A'}{R}}
                \\ \text{\( n \) is neutral}}
                {\quotmR{\Gamma}{n}}
    \]}
  \item \( \eqtmR{\ell}{\Gamma}{t}{u}{A} \) if there are normal forms \( t', u' \) such that
    \( \redT{\Gamma}{t}{t'}{\Quo{A'}{R}} \) and \( \redT{\Gamma}{u}{u'}{\Quo{A'}{R}} \), and
    \( \quoeqR{\Gamma}{t'}{u'} \), which is inductively defined by
{\small
    \[
      \inferrule{\eqtmR{\ell}{\Gamma}{t}{u}{A'}}
                {\quoeqR{\Gamma}{\quo{t}}{\quo{u}}}
      \quad
      \inferrule{\eqtm{\Gamma}{n}{m}{\Quo{A'}{R}}
                \\ \text{\( n, m \) are neutral}}
                {\quoeqR{\Gamma}{n}{m}}
    \]}
  \end{itemize}
  %
% Then, we also need to show that the proof of the fundamental lemma can be extended to handle the
% new typing rules, as well as the new cases of logical relation. For this, we need to show five
% lemmas \lp{Do we really need to spell them out?}
Then, the proof of the fundamental lemma can be extended to handle the
new typing rules, as well as the new cases of logical relation.


\paragraph{Interpretation in the Model}

The type \( \Setoid \) of setoids is naturally closed under quotients. Indeed, given a setoid
\( A \), providing \( A \) with a relation \( \tm{R}{\sFun{A \times A}{\metasProp}} \) that is
reflexive, symmetric and transitive is exactly the same thing as defining an equivalence relation on
the carrier set of \( A \) that extends setoidal equality. Then, \( A/R \) is simply defined
as a setoid having \( A \) for its carrier set, and the relation \( R \) as its setoidal equality.
One can then easily show that \( A/R \) satisfies the universal property of a mathematical
quotient, which tells us that this construction is the right candidate to interpret our quotient
types.

However, our universe hierarchy \( \sU_i \) is not closed under quotients, since its elements are
inductively built from \( \metanat \), universes and function types. Therefore, we need to modify
our inductive-recursive definition to account for them. In the definition of \( \sU_i \), we add
a constructor
{\small
\[
\begin{array}{rcl}
  \codequo & : & (\tm{A}{\sU_i}) \\
    & \rightarrow & (\tm{R}{\sFun{\el\ A}{\sFun{\el\ A}{\ssProp_i}}}) \\
    & \rightarrow & (\tm{R_r}{\sFun{(x : \el\ A)}{\val\ (R\ x\ x)}}) \\
    & \rightarrow & (\tm{R_s}{\sFun{(x\ y : \el\ A)}{\Fun{\val\ (R\ x\ y)}{\val\ (R\ y\ x)}}}) \\
    & \rightarrow & (\tm{R_t}{\sFun{(x\ y\ z : \el\ A)}{\Fun{\val\ (R\ x\ y)}{\Fun{\val\ (R\ y\ z)}{\val\ (R\ x\ z)}}}}) \\
    & \rightarrow & \sU_i
\end{array}
\]
}
%
In the definition of \( \el \), our new constructor is handled as follows:
{\small
\[
  \el\ (\codequo\ A\ R\ R_r\ R_s\ R_t)
  \ := \
  (\el\ A)/(\lambda\, x\, y .\, \val\ (R\ x\ y))
\]
}
In the definition of the setoidal equality on \( \sU_i \):
{\small
\[
  \sEq{\codequo\ A\ R\ R_r\ R_s\ R_t}{\codequo\ A'\ R'\ R'_r\ R'_s\ R'_t}
  \ := \
  (\tm{e}{\sEq{A}{A'}}) \times (\sFun{(\tm{x\ y}{A})}{\sEq{R\ x\ y}{R'\ (\eleq\ e\ x)\ (\eleq\ e\ y)}})
\]
}
and all of the other cases that involve \( \codequo \) and a different constructor reduce to
\( \metaempty \). We also need to update the definition of \( \eleq \):
{\small
\[
  \eleq\ (\codequo\ A\ R\ R_r\ R_s\ R_t)\ (\codequo\ A'\ R'\ R'_r\ R'_s\ R'_t)\ e
  \ := \
  (\lambda x.\eleq\ A\ A'\ e.1\ x\ ,\ \lambda x.\eleq^{-1}\ A\ A'\ e.1\ x)
\]
}
Finally, the reader can check that we can extend the proofs of reflexivity, symmetry and
transitivity for \( \approx \) on \( \sU_i \), as well as the proofs that \( \eleq \)
is compatible with reflexivity and transitivity. This defines a new universe that is closed
under quotient type formation.
%
We let the reader check that the interpretation of the syntax that
defines quotients can be done in this extended universe \( \sU_i \).

\subsection{Id Types}

The reader may wonder if \SetoidTT extends Martin-Löf type theory
with inductive types: that is, whether a judgment of MLTT is a
judgment in the proof-relevant fragment of \SetoidTT. Our rules
handle the universe hierarchy, dependent products, the natural
numbers in the exact same way. But there are some difficulties with
Martin-Löf identity type inductive equality, and more generally with
indexed inductive types.

The first idea that might come to the reader's mind is to use the proof equality types of \SetoidTT
to interpret the \( I \)-types of MLTT. Sure enough, proof irrelevance will provide us with more
definitional equalities than what we require.
% , but this is not a problem at first glance.
However, we
need to explain how we interpret the \( J \) eliminator for proof-relevant predicates.

It is not too hard to design a term that satisfies the correct typing
rule, for instance, the term
{\small
\[
  \inferrule{\tytm{\Gamma}{A}{\varType_i}
            \\ \tytm{\Gamma}{t}{A}
            \\ \tytm{\Gamma}{B}{\Depfun{A}{\Fun{\Obseq[A]{t}{x}}{\varType_j}}}
            \\ \tytm{\Gamma}{b}{B\ t\ \refl{t}{A}}
            \\ \tytm{\Gamma}{t'}{A}
            \\ \tytm{\Gamma}{e}{\Obseq[A]{t}{t'}}}
            {\tytm{\Gamma}{\cast{B\ t\ \refl{t}{A}}{B\ t'\
  e}{\eqJ{A}{t}{u}{t'}{e}}{b}}{B\ t'\ e}}
\]}
where
{\small
$$\eqJ{A}{t}{B}{u}{t'}{e} :=
\transport{A}{t}{\lam{A}{\lam[e']{\Obseq{t}{x}}{\Obseq{B\ t\ \refl{A}{t}}{B\ x\ e'}}}}{\refl{B\ t\ \refl{A}{t}}{}}{t'}{e}.$$
}
But the computational behavior is not preserved: in general, this term will not reduce to \( u \)
when we substitute \( t' = t \) and \( e = \refl{t}{A} \). It might do when \( B \) is a closed
term, but it certainly wil not if \( B \) is a neutral term. More generally, there is no hope to
interpret \( I \)-types as proof-irrelevant types: \( I \)-types compute by doing reduction and
pattern-matching on the equality proof, which cannot happen in a proof-irrelevant context.

A very similar problem was encountered by \sidecitet{cubicaltt} in Cubical Type Theory with
equality defined using the $\CubicalPath$ type, and solved
by \sidecitet{Swan_2016}.
Following his ideas, we introduce Id-types:
{\small
\[
  \inferrule{\tytm{\Gamma}{A}{\varType_i}
            \\ \tytm{\Gamma}{t}{A}
            \\ \tytm{\Gamma}{u}{A}}
            {\tytm{\Gamma}{\Id{A}{t}{u}}{\varType_i}}
  \quad
  \inferrule{\tytm{\Gamma}{A}{\varType_i}
            \\ \tytm{\Gamma}{t}{A}}
            {\tytm{\Gamma}{\id{t}}{\Id{A}{t}{t}}}
  \quad
  \inferrule{\tytm{\Gamma}{e,e'}{\Id{A}{t}{u}}}
            {\red{\Gamma}{\Obseq[\Id{A}{t}{u}]{e}{e'}}{\Unit}{\sProp[i]}}
\]
\[
  \inferrule{\tytm{\Gamma}{A}{\varType_i}
            \\ \tytm{\Gamma}{t}{A}
            \\ \tytm{\Gamma}{B}{\Depfun{A}{\Fun{\Id{A}{t}{x}}{\Type[j]{}}}}
            \\ \tytm{\Gamma}{u}{B\ t\ \id{t}}
            \\ \tytm{\Gamma}{t'}{A}
            \\ \tytm{\Gamma}{e}{\Id{A}{t}{t'}}}
            {\tytm{\Gamma}{\J{A}{t}{B}{u}{t'}{e}}{B\ t'\ e}}
\]
\[
  \inferrule{\tytm{\Gamma}{A}{\varType_i}
            \\ \tytm{\Gamma}{t}{A}
            \\ \tytm{\Gamma}{B}{\Depfun{A}{\Fun{\Id{A}{t}{x}}{\Type[j]{}}}}
            \\ \tytm{\Gamma}{b}{B\ t\ \id{t}}}
            {\red{\Gamma}{\J{A}{t}{B}{b}{t}{\id{t}}}{u}{B\ t\ \id{t}}}
          \]
          }
These rules mimic the behavior of inductive \( I \)-types, quotiented so they contain
only one inhabitant up to propositional equality. Observational
equality between two identity types is defined as equality of the
telescopes of arguments, as for the quotient type:
%
{\small
\[
  \inferrule{\tytm{\Gamma}{A}{\varType_i}
            \\ \tytm{\Gamma}{t}{A}
            \\ \tytm{\Gamma}{u}{A}
            \\ \tytm{\Gamma}{A'}{\varType_i}
            \\ \tytm{\Gamma}{t'}{A'}
            \\ \tytm{\Gamma}{u'}{A'}}
            {\redmultiline{\Gamma}
              {\Obseq[\varType]{\Id{A}{t}{u}}{\Id{A'}{t'}{u'}}}
              {\Exists[e]{\Obseq[\varType]{A}{A'}}{\Prod{\Obseq[A']{\cast{A}{A'}{e}{t}}{t'}}{\Obseq[A']{\cast{A}{A'}{e}{u}}{u'}}}}
              {\sProp[j]}}
          \]
          }
%
We may hope to define computation
rules for cast by simply reducing the equality proof to a weak head normal form, and then
commuting cast with the head constructor like we did for other positive types. However, we quickly
run into a problem:
{\small
\[
  \inferrule{\tytm{\Gamma}{A, A'}{\varType_i}
            \\ \tytm{\Gamma}{t}{A}
            \\ \tytm{\Gamma}{t,u}{A'}
            \\ \tytm{\Gamma}{e}{\Obseq{\Id{A}{t}{t}}{\Id{A'}{t'}{u'}}}}
            {\red{\Gamma}{\cast{\Id{A}{t}{t}}{\Id{A'}{t'}{u'}}{e}{\id{t}}}{\ ?}{\Id{A'}{t'}{u'}}}
\]}
We cannot reduce this term to $\metaop{Idrefl}$, because \( t' \) and \( u' \)
are not convertible in general---\( e \) only provides us with a propositional equality \( \Obseq[A']{t'}{u'} \). In order to fix
this, we add a term $\idpath{e} $ that turns any inhabitant of \(
e : \Obseq[A]{t}{u} \) into an inhabitant of
\( \Id{A}{t}{u} \).
{\small
\[
  \inferrule{\tytm{\Gamma}{A}{\varType_i}
            \\ \tytm{\Gamma}{t}{A}
            \\ \tytm{\Gamma}{u}{A}
            \\ \tytm{\Gamma}{e}{\Obseq[A]{t}{u}}}
            {\tytm{\Gamma}{\idpath{e}}{\Id{A}{t}{u}}}
\]}
%
Then, the computation rule for cast on $\id{t}$ can be defined as:
%
{\small
\[
  \inferrule{\tytm{\Gamma}{A, A'}{\varType_i}
            \\ \tytm{\Gamma}{t}{A}
            \\ \tytm{\Gamma}{t',u'}{A'}
            \\ \tytm{\Gamma}{e}{\Obseq{\Id{A}{t}{t}}{\Id{A'}{t'}{u'}}}}
            {\redmultiline{\Gamma}
              {\cast{\Id{A}{t}{t}}{\Id{A'}{t'}{u'}}{e}{\id{t}}}
              {\idpath{\transitivity{\sym{\fst{\snd{e}}}}{\snd{\snd{e}}}}}
              {\Id{A'}{t'}{u'}}}
\]}
We also need to account for this additional constructor in reduction rules:
{\small
\[
  \inferrule{\tytm{\Gamma}{A}{\varType_i}
            \\ \tytm{\Gamma}{t}{A}
            \\ \tytm{\Gamma}{B}{\Depfun{A}{\Fun{\Id{A}{t}{x}}{\Type[j]{}}}}
            \\ \tytm{\Gamma}{b}{B\ t\ \id{t}}
            \\ \tytm{\Gamma}{t'}{A}
            \\ \tytm{\Gamma}{e}{\Obseq[A]{t}{t'}}}
            {\redmultiline{\Gamma}
              {\J{A}{t}{B}{b}{t'}{\idpath{e}}}
              {\cast{B\ t\ \id{t}}{B\ t'\ \idpath{e}}{\eqJ{A}{t}{B}{u}{t'}{e} }{b}}
              {B\ t'\ \idpath{e}}}
\]
\[
  \inferrule{\tytm{\Gamma}{A, A'}{\varType_i}
            \\ \tytm{\Gamma}{t, u}{A}
            \\ \tytm{\Gamma}{e}{\Obseq[A]{t}{u}}
            \\ \tytm{\Gamma}{t',u'}{A'}
            \\\tytm{\Gamma}{e'}{\Obseq{\Id{A}{t}{t}}{\Id{A'}{t'}{u'}}}}
            {\redmultiline{\Gamma}
              {\cast{\Id{A}{t}{u}}{\Id{A'}{t'}{u'}}{e'}{\idpath{e}}}
              {\idpath{\transitivity{\transitivity{\sym{\fst{\snd{e'}}}}{\ap{(\cast{A}{A'}{\fst{e'}}{-})}{e}}}{\snd{\snd{e'}}}}}
              {\Id{A'}{t'}{u'}}}
\]}
Along with congruence rules and reduction of the scrutinee of \( J \) and cast, these form the
rules for Id types.

\paragraph{Relation with Setoid Equality}

The introduction rule for $\metaop{IdPath}$ proves that \( \Obseq[A]{t}{u} \) imples \( \Id{A}{t}{u} \).
Conversely, if we have an inhabitant of \( \Id{A}{t}{u} \), we can get a proof of
\( \Obseq[A]{t}{u} \) by combining \( \metaop{J} \) to reduce it to the case \( u = t \), and
\( \refl{t}{A} \) to inhabit \( \Obseq[A]{t}{t} \). Since both types are also contractible
(for both notions of equality), they are equivalent.

\paragraph{Reducibility Proof}

In order to fit identity types in the normalization proof, we add another case to the logical relation:
{\small
  \[
  \inferrule{\redT{\Gamma}{A}{\Id{A'}{t}{u}}{\varType_i}
            \\ \tytyR{\ell}{\Gamma}{A'}
            \\ \tytmR{\ell}{\Gamma}{t}{A'}
            \\ \tytmR{\ell}{\Gamma}{u}{A'}}
            {\tytyR{\ell}{\Gamma}{A}}
\]}
When $A$ is reducible to an identity type, we define:
\begin{itemize}
  \item \( \eqtyR{\ell}{\Gamma}{A}{B} \) if there are terms \( B', t', u' \) such that
    \begin{itemize}
      \item \( \redT{\Gamma}{B}{\Id{B'}{t'}{u'}}{\varType_i} \)
      \item \( \eqtyR{\ell}{\Gamma}{A'}{B'} \)
      \item \( \eqtmR{\ell}{\Gamma}{t}{t'}{A'} \)
      \item \( \eqtmR{\ell}{\Gamma}{u}{u'}{A'} \).
    \end{itemize}
  \item \( \tytmR{\ell}{\Gamma}{e}{A} \) if there is a normal form \( e' \) such that
    \( \redT{\Gamma}{e}{e'}{\Id{A'}{t}{u}} \) and \( \idtmR{\Gamma}{e'} \), which is defined by
{\small
    \[
      \inferrule{ }
                {\idtmR{\Gamma}{\id{t}}}
      \quad
      \inferrule{\tytm{\Gamma}{e}{\Obseq[A']{t}{u}}}
                {\idtmR{\Gamma}{\idpath{e}}}
      \quad
      \inferrule{\tytm{\Gamma}{n}{\Id{A'}{t}{u}}
                \\ \text{\( n \) is neutral}}
                {\idtmR{\Gamma}{n}}
    \]}
  \item \( \eqtmR{\ell}{\Gamma}{e}{f}{A} \) if there are normal forms \( e', f' \) such that
    \( \redT{\Gamma}{e}{e'}{\Id{A'}{t}{u}} \) and \( \redT{\Gamma}{f}{f'}{\Id{A'}{t}{u}} \), and
    \( \ideqR{\Gamma}{e'}{f'} \), which is inductively defined by
{\small
    \[
      \inferrule{ }
                {\ideqR{\Gamma}{\id{t}}{\id{t}}}
      \quad
      \inferrule{\tytm{\Gamma}{e, f}{\Obseq[A']{t}{u}}}
                {\ideqR{\Gamma}{\idpath{e}}{\idpath{f}}}
    \]
    \[
      \inferrule{\eqtm{\Gamma}{n}{m}{\Id{A'}{t}{u}}
                \\ \text{\( n, m \) are neutral}}
                {\ideqR{\Gamma}{n}{m}}
    \]}
  \end{itemize}

Again, the proof of the fundamental lemma can be extended to handle the
new typing rules, as well as the new cases of the logical relation.

% \subsection{General Inductive Types}

% \lp{Something about reducing general inductive types to \( \Sigma \), \( \Nat \) and Id.}
% \[
% \inferrule{\tytm{\Gamma}{A}{\varType_i} \\ \tytm{\extctx{\Gamma}{A}}{B}{\varType_j}}{\tytm{\Gamma}{\Sig{A}{B}}{\varType_k}} {\scriptstyle \substack{i \le k \\ j \le k}}
% \]

\subsection{Box and Squash}

Box types embed the proof-irrelevant types into the proof-relevant world.
They are useful for a number of constructions: for instance, from a type \( \tm{A}{\varType_i} \) and a
proof-irrelevant predicate \( \tm{P}{\Fun{A}{\sProp[i]}} \), one can build a \emph{subset type}
\( \tm{\Sig{A}{\Boxt{(P\ x)}}}{\varType_i} \) whose inhabitants come with a proof of \( P \), but
retain the computational behavior of inhabitants of \( A \).

Another typical use of $\Boxt{}$ is to define a singleton
  type on which it is possible to reason about equality. Indeed, although
  $\Unit$ has only one inhabitant up-to conversion, it is not possible
  to state this internally as equality on propositions is not defined.
  %
  However, one can state contractibility of the type $\Boxt{\Unit}$ and prove
  it, as it lives in $\varType$.

Box types can be defined as:

{\small
\[
  \inferrule{\tytm{\Gamma}{A}{\sProp[i]}}
            {\tytm{\Gamma}{\Boxt{A}}{\varType_i}}
  \quad
  \inferrule{\tytm{\Gamma}{A}{\sProp[i]}
            \\ \tytm{\Gamma}{t}{A}}
            {\tytm{\Gamma}{\boxt{t}}{\Boxt{A}}}
  \quad
  \inferrule{\tytm{\Gamma}{t,u}{\Boxt{A}}}
            {\red{\Gamma}{\Obseq[\Boxt{A}]{t}{u}}{\Unit}{\sProp[i]}}
  \quad
  \inferrule{\tytm{\Gamma}{A}{\sProp[i]}
            \\ \tytm{\Gamma}{t}{\Boxt{A}}}
            {\tytm{\Gamma}{\unboxt{t}}{A}}
\]
\[
  \inferrule{\tytm{\Gamma}{A,B}{\sProp[i]}}
            {\red{\Gamma}
              {\Obseq[\varType]{\Boxt{A}}{\Boxt{B}}}
              {\Obseq[\sProp]{A}{B}}
              {\sProp[j]}}
  \quad
  \inferrule{\tytm{\Gamma}{A, B}{\sProp[i]}
            \\ \tytm{\Gamma}{t}{A}
            \\ \tytm{\Gamma}{e}{\Obseq[\varType]{\Boxt{A}}{\Boxt{B}}}}
            {\red{\Gamma}{\cast{\Boxt{A}}{\Boxt{B}}{e}{\boxt{t}}}{\boxt{\cast{A}{B}{e}{t}}}{\Boxt{B}}}
\]
}

Conversely, Squash types embed the proof-relevant world into the proof-irrelevant
world.
{\small
\[
  \inferrule{\tytm{\Gamma}{A}{\varType_i}}
            {\tytm{\Gamma}{\Squash{A}}{\sProp[i]}}
  \quad
  \inferrule{\tytm{\Gamma}{A}{\varType_i}
            \\ \tytm{\Gamma}{t}{A}}
            {\tytm{\Gamma}{\squash{t}}{\Squash{A}}}
\]
\[
  \inferrule{\tytm{\Gamma}{A}{\varType_i}
            \\ \tytm{\Gamma}{P}{\Fun{\Squash{A}}{\sProp[j]}}
            \\ \tytm{\Gamma}{t_A}{\Depfun{A}{P\ \squash{x}}}
            \\ \tytm{\Gamma}{t}{\Squash{A}}}
            {\tytm{\Gamma}{\unsquash{P}{t_A}{t}}{P\ t}}
\]}
It is not very difficult to check that we can extend the logical relation,
as well as \cref{reducibilitycast,reducibilityidU} so that they
handle Box types and Squash types.
%
Likewise, we can extend the setoid model and its interpretation function, and obtain
that these additions preserve the metatheoretical properties of \SetoidTT.

% \section{Implementing Observational Equality with Rewrite Rules}
% \label{sec:impl-observ-equal}

% \sidecite{taming_of_the_rew}

% \subsection{Application: Monadic Laws without Axioms}

% Example of effectful monads and their laws (states require fun ext and exceptions requires UIP).

\subsection{Other Standard Inductive Types}
\label{sec:sheph-stand-induct}

So far, we have explained how to integrate integers and the
  identity type to \SetoidTT, but it is not difficult to integrate
  $\Sigma$-types as well. The rules for $\Sigma$-types are
  similar to the rules for $\exists$-types, except for the fact that
  the types live in $\varType$, which means that we also need to define
  equality and cast on $\Sigma$-types.
  We note $(a;b)$ for pairs in $\Sigma$-types to distinguish them from
  pairs in $\exists$-types.
%
\begin{center}
  \begin{small}
    \begin{mathpar}
  \inferrule[Eq-Pair]{\tytm{\Gamma}{a}{A} \\ \tytm{\Gamma}{a'}{A} \\
    \tytm{\Gamma}{b}{\subst{B}{a}}
            \\ \tytm{\Gamma}{b'}{\subst{B}{a'}} \\ e' := \ap{B}{e}} %\transport{}{a}{\lam{A}{B}}{a'}{e}{b}{}}
            {\red{\Gamma}{\Obseq[\Sigma A B]{(a;b)}{(a';b')}}{\Exists[e]{\Obseq[A]{a}{a'}}
                {\Obseq[B]{\cast{\subst{B}{a}}{\subst{B}{a'}}{e'}{b}}{b'}}}{\sProp[i]}}
          \ilabel{inferrule:eq-pair}
\and
  \inferrule[Eq-$\Sigma$]{\tytm{\Gamma}{A,A'}{\Type[i]{}}
            % \\ \tytm{\Gamma}{A'}{\Type[i]{}}
            \\ \tytm{\extctx{\Gamma}{A}}{B}{\Type[j]{'}}
            \\ \tytm{\extctx{\Gamma}{A'}}{B'}{\Type[j]{'}}
            \\ a := \cast{A'}{A}{\sym{e}}{a'}}
            {\redmultiline{\Gamma}
              {\Obseq[{\varType_k}]{\Depsum{A}{B}}{\Depsum{A'}{B'}}}
              {\Exists[e]{\Obseq[{\varType_i}]{A}{A'}}{\Depfun[a']{A'}{\Obseq[{\varType_j}]{\subst{B}{a}}{\subst{B'}{a'}}}}}{\sProp[i]}}
            {\scriptstyle \substack{i \le k \\ j \le k}}
            \ilabel{inferrule:eq-sigma}
\and
  \inferrule[Cast-$\Sigma$]{\tytm{\Gamma}{e}{\Obseq[\varType]{\Depsum{A}{B}}{\Depsum{A'}{B'}}}
    \\ \tytm{\Gamma}{a}{A} \\ \tytm{\Gamma}{b}{\subst{B}{a}}
    \\ a' := \cast{A}{A'}{\fst{e}}{a}}
            {\redmultiline{\Gamma}
              {\cast{\Depsum{A}{B}}{\Depsum{A'}{B'}}{e}{(a;b)}}
              {(a'; \cast{\subst{B}{a}}{\subst{B}{a'}}{\snd{e}}{b}) }
              {\Depsum{A'}{B'}}}
             \ilabel{inferrule:cast-sigma}
           \end{mathpar}
           \end{small}
  % \caption{\SetoidTT Reduction Rules (except substitutions)}
  % \label{fig:SetoidTT-reduction}
\end{center}
%
We conjecture that the reduction of equality and cast for
any indexed inductive types as defined in \CIC can be described, although we
leave the general construction for future work.
