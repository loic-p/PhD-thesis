\setchapterimage[6cm]{seaside}
\setchapterpreamble[u]{\margintoc}
\chapter{Intensional Type Theory}
\labch{layout}

Dependent type theories and in particular \MLTT, as originally
developed by~\sidecitet{MARTINLOF197573}, provide an adequate framework
for developing constructive mathematics and certifying software.
%
A core aspect of \MLTT is the coexistence of two distinct notions of
equality: a \emph{definitional} equality that records the equations
automated by the system, and a \emph{propositional} equality that is
a type internal to the system and thus can be used to do equational
reasoning.
%
However, the propositional equality of \MLTT lacks some extensionality
principles that pervade mathematical reasoning, such as function
extensionality (funext). Since they are generally considered desirable, these
principles are sometimes added as axioms, but doing so results in a system
with weaker computational properties.

Throughout the years, several options
to obtain a more extensional version of propositional equality while preserving
computation have been explored. The most successful lines of work can be roughly divided into two
groups: the ones using an observational equality, and the ones using a
cubical equality.
%
Both notions build on the following fundamental idea: in order to obtain
sufficient extensionality principles, the behavior of equality in a given
type should be explicitly specified for every type instead of using a single
definition that is parametric over the types.

The work on observational equality originated from the study of
the setoid model of type theory \sidecite{hofmann95,altenkirch99}, and
a first attempt at a proper type theory that supports an observational
equality satisfying funext has been been proposed in~\sidecite{altenkirchAl:plpv2007}.
%
In these systems, observational equality equips every type with a setoid
structure. This setoidal equality satisfies the uniqueness of identity proofs (UIP),
which states that all proofs of an equality are equal; in other words,
equality is proof-irrelevant.

The other line of work is more recent and takes its roots in the
formulation of the univalence axiom~\sidecite{kapulkin2012simplicial,hottbook},
which gives a new meaning to the equality between types: two types are
equal when they equivalent. This can be understood as an extensionality
principle for the universe of types.
%
In their search for a computational interpretation of
univalence, \sidecitet{cubicaltt} developed a notion of cubical equality,
which satisfies funext and univalence, but is incompatible with UIP.

Thus, observational equality and cubical equality are two diverging
directions for providing propositional equality with more extensionality.
\sidecitet{hts-sota,Altenkirch2016ExtendingHT,Capriotti17} advocate that the
two solutions are actually complementary and can be integrated to
a single system with two universe hierarchies, one that satisfies
univalence and the other that satisfies UIP.
%
While cubical type theory has been thoroughly investigated and even
implemented in the \Agda proof assistant~\sidecitet{cubical-agda},
observational equality has not reached a comparable level of maturity.
%
Recent attempts to design a type theory based on observational equality
are either lacking an algorithm for type checking~\sidecitet{sterling_et_al:LIPIcs:2019:10538}, or restricted to a single
universe and having computational properties only up to a
conjecture~\sidecitet{Altenkirch2019}. Indeed, the latter relies on computation
in an enriched version of \MLTT that features a universe of definitionally
proof-irrelevant types (noted hereafter $\sProp$) as recently proposed
by~\sidecitet{gilbert:hal-01859964}, along with a proof-irrelevant identity type
that supports a strong eliminator. This theory has not been justified yet,
and it has even been shown not to be normalizing in presence of
impredicativity~\sidecitet{lmcs:6606}.

In this paper, we define \SetoidTT, the first extension of \MLTT +
$\sProp$ with an observational equality that satisfies UIP, funext
and propext, and supports quotient types and countably many universes
---with a proof of normalization and canonicity formalized in \Agda.\footnote{The formalization is available at
  \url{https://github.com/CoqHott/logrel-mltt/}, references to a
  particular file are done using \refAgda{}{myfile} which
  directly points to the corresponding file on github.}
%
Firstly, we remark that this theory can only be derived in a system with
some level of cumulativity, as it is required for certain
computation rules to be well-typed. This makes explicit some difficulties
that do not show up in previous works.
%
Second, we remark that our version of observational equality can be equipped
with two elimination principles: a notion of type cast (or coercion) for
proof-relevant types, and the standard eliminator of \MLTT for
proof-irrelevant types, thus making all the setoidal structure
derivable from the standard eliminator.
%
Funext and propext are obtained by specifying the right
computation rules for observational equality whereas UIP is obtained
for free by interpreting observational equality in $\sProp$.

The proof of normalization and canonicity is based on the use
of logical relations defined in \Agda using induction-recursion as
initially developed by \sidecitet{Abel:POPL2018} and later extended to
$\sProp$ by~\sidecitet{gilbert:hal-01859964}.
%
Compared to previous work on formalized normalization proofs, we have added the support
for a cumulative hierarchy with two universes, and we make a clear distinction
between inhabitants of proof-irrelevant types, which have no computational
behavior, and inhabitants of proof-relevant types, which do.
%
This key change allows us to add new principles in $\sProp$ without
having to supply them with a computational behavior, trivializing for
instance the management of higher coherences of the cubical interpretation.
%
In counterpart for this added flexibility, normalization does not directly
imply canonicity anymore, and a separate proof of consistency of the
theory is required to derive canonicity.
%
This proof is done by defining a model for \SetoidTT, but as
consistency is the only consequence that we need from the existence of
a model, the model can be defined in an extensional setting.

To illustrate the simplicity of extending \MLTT+$\sProp$ to \SetoidTT,
we have implemented a simple version in
\Agda using rewrite rules, as recently introduced
by~\sidecitet{taming_of_the_rew} (file \refAgdaRoot{setoid\_rr}).

\section{Related work}

Compared to~\sidecite{altenkirchAl:plpv2007}, the most important ingredient in \SetoidTT is the
use of definitional proof-irrelevance.
%
This added flexibility in computations allows our recursors to enjoy proper computational
behavior on open terms, and it also lets us seamlessly treat universe hierarchies.
%
Moreover, the normalization proof for OTT relies on a normalization conjecture for a different
theory, unlike the normalization proof for \SetoidTT.

In~\sidecite{Altenkirch2019}, the authors define a setoid model in \( \MLTT+\sProp \). Then, they
interpret a version of \MLTT with proof-irrelevant identity types that support propext and
funext in their model, thereby providing a computational interpretation of these principles.
%
However, handling universes in their model requires some additions to \( \MLTT+\sProp \), and
the resulting theory is only conjectured to be normalizing.
%
In contrast to this, \SetoidTT is a full-fledged type theory and does not require any external
model to compute.

Compared to XTT~\sidecite{sterling_et_al:LIPIcs:2019:10538}, the strengths of \SetoidTT are a
normalization strategy that exhibits canonicity, as well a full proof that conversion and typing
are decidable. These properties allow us to present a concrete implementation of our system
in a proof assistant.
%
In XTT however, Sterling\etal show that typing cannot be decidable, as there is no way to deduce
\( \Obseq{A}{A'} \) and \( \Obseq{B}{B'} \) from a proof of \( \Obseq{A \times B}{A' \times B'} \).
%
In order to fix this shortcoming, they suggest adding a ``typecase'' operator, but
argue against it since it forces the universe to be closed, thereby severely constraining the
possible semantics.
%
In \SetoidTT, we obtain the injectivity of type contructors from the behavior of observational
equality in the universe. These rules somewhat constrain the semantics---for instance, we cannot
interpret \SetoidTT in set theory using a Grothendieck universe as the interpretation of \( \Type \)---but our universe remains open to the addition of arbitrary types.


\section{the two problems being solved}

Intensional computation with extensional principles

Normalization of proof-irrelevant impredicative universe with elimination of false and equality

\section{history of setoids}

Hofmann : Extensional concepts in intensional type theory (PhD thesis, 95)

First model of ETT in ITT. Does not support universes because no proof-relevant equality.

Altenkirch : Extensional Equality in Intensional Type Theory (LICS99)

Full model of ETT in ITT + sProp, using CwF of setoids.

Altenkirch, McBride : 

Altenkirch, McBride : Observational Equality, Now! (PLPV07)

Normalization ``proof'', canonicity from consistency, proof-irrelevant axioms 
are fine, heterogeneous equality.

McBride et al : Epigram 2

Altenkirch, Boulier, Kaposi, Tabareau :

Sterling :
