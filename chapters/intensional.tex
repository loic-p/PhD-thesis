\setchapterimage[6cm]{seaside}
\setchapterpreamble[u]{\margintoc}
\chapter{Introduction in English}
\labch{intro_observational}

\section{Related work}

Compared to~\sidecite{altenkirchAl:plpv2007}, the most important ingredient in \SetoidTT is the
use of definitional proof-irrelevance.
%
This added flexibility in computations allows our recursors to enjoy proper computational
behavior on open terms, and it also lets us seamlessly treat universe hierarchies.
%
Moreover, the normalization proof for OTT relies on a normalization conjecture for a different
theory, unlike the normalization proof for \SetoidTT.

In~\sidecite{Altenkirch2019}, the authors define a setoid model in \( \MLTT+\sProp \). Then, they
interpret a version of \MLTT with proof-irrelevant identity types that support propext and
funext in their model, thereby providing a computational interpretation of these principles.
%
However, handling universes in their model requires some additions to \( \MLTT+\sProp \), and
the resulting theory is only conjectured to be normalizing.
%
In contrast to this, \SetoidTT is a full-fledged type theory and does not require any external
model to compute.

Compared to XTT~\sidecite{sterling_et_al:LIPIcs:2019:10538}, the strengths of \SetoidTT are a
normalization strategy that exhibits canonicity, as well a full proof that conversion and typing
are decidable. These properties allow us to present a concrete implementation of our system
in a proof assistant.
%
In XTT however, Sterling\etal show that typing cannot be decidable, as there is no way to deduce
\( \Obseq{A}{A'} \) and \( \Obseq{B}{B'} \) from a proof of \( \Obseq{A \times B}{A' \times B'} \).
%
In order to fix this shortcoming, they suggest adding a ``typecase'' operator, but
argue against it since it forces the universe to be closed, thereby severely constraining the
possible semantics.
%
In \SetoidTT, we obtain the injectivity of type contructors from the behavior of observational
equality in the universe. These rules somewhat constrain the semantics---for instance, we cannot
interpret \SetoidTT in set theory using a Grothendieck universe as the interpretation of \( \Type \)---but our universe remains open to the addition of arbitrary types.

\section{to put somewhere}

Allais and McBride on computation on refl

Bob atkey on observational equality that does not reduce

impossible to have canonicity for equality

inductive equality is predicative

explain the n+4

\sideremark{The \Coq proof assistant extends to the base calculus
of constructions
Predicative Calculus of Cumulative Inductive Constructions \cite{pcuic}}

Mixing type coercion operators and impredicative propositions is well known to 
be a minefield:
% % 
% in his seminal work on System F, Girard already shows that adding an operator 
% \( {J : \forall A\ B\ .\ A \to B} \) with the reduction rule
% \( {J\ A\ A\ t \Rightarrow t} \) breaks normalization \sidecite{girard72}.
in recent years, Coquand and Abel showed that the interaction of the two features 
breaks normalization in the type theory of \Lean \sidecite{lmcs:6606}.
% 
Still, they leave an open question: is there a way to decide conversion without

% And does the same argument applies to the observational equality of \SetoidCC, 
% which computes in a different way?
 
% As we will see, the answers to those questions are respectively ``no'' and 
% ``yes''.