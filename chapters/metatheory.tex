\setchapterimage[6cm]{seaside}
\setchapterpreamble[u]{\margintoc}
\chapter{Meta-theory of TTobs}
\labch{metatheory}

In this section, we will prove these four properties for \SetoidCC.
%
We first present our extension of the
logical relations framework developed by \sidecitet{Abel:POPL2018} and
later extended to $\sProp$ by~\sidecitet{gilbert:hal-01859964}. The
framework provides a proof of normalization and canonicity from
consistency, as well as decidability of conversion and thus decidability of
type checking.
%
It has been formalized in the \Agda development.
%
Then, to get consistency and therefore canonicity, we develop a model of
\SetoidCC in a constructive set theory that supports inductive-recursive definitions and one
Grothendieck universe.


\section{A logical relation with an Impredicative universe}
\label{sec:logic-relat-with}

\begin{figure*}
    \[
  \begin{array}{lr}
    \tytyR{\ell}{\Gamma}{A}{\Type{}} & \text{A is a reducible type of sort \( s \) in context \( \Gamma \)} \\
    \eqtyR{\ell}{\Gamma}{A}{B}{\Type{}} & \text{A and B are reducibly equal types of sort \( s \) in context \( \Gamma \)} \\
    \tytmR{\ell}{\Gamma}{t}{A}{\Type{}} & \text{t is a reducible term of type \( A \), which has sort \( s \) in context \( \Gamma \)} \\
    \eqtmR{\ell}{\Gamma}{t}{u}{A}{\Type{}} & \qquad\text{t and u are reducibly equal terms of type \( A \), which has sort \( s \) in context \( \Gamma \)}
  \end{array}
\]
  \caption{The four judgments of the logical relation}
  \label{fig:log-rel-judgments}
\end{figure*}

In this section, we define a logical relation to show
normalization and decidability of conversion for \SetoidCC, following
\sidecitet{Abel:POPL2018}.

Normalization of dependent type theories is a subtle matter:
if one tries to prove it by naive induction on the typing derivations,
one quickly gets stuck on the case of the application rule because
we cannot deduce normalization of \( t~u \) from normalization of
$t$ and $u$.
%
The so-called ``logical relations'' is a standard family of techniques 
used to strengthen the induction hypothesis, so that the proof by induction 
on the typeing derivation goes through.
%
Concretely speaking, the strengthened induction hypothesis collects information 
on types and terms that thoroughly characterizes their behaviour with respect 
to substitution and reduction.

The base ingredient of the logical relation is \emph{reducibility}, a set
of four predicates which mirror the typing judgments of \SetoidCC 
(cf \cref{fig:log-rel-judgments}), defined on a type-by-type basis.
% 
For instance, an term is deemed to be a reducible inhabitant of a \( \Pi \)-type 
if it normalizes to a weak-head normal form, and it sends reducible 
arguments to reducible images.
% 
However, reducibility is not quite strong enough for the induction to go 
through. We must first close it under substitutions, and only after this
can we prove the \emph{fundamental lemma} which shows that any well-typed 
term is reducible by induction on the typing derivation.

For \MLTT and \SetoidCC, the logical relation is indexed by a level
\( \ell \), which reflects the predicative nature of the universe
hierarchy: reducibility is first defined at level $0$ to characterize
terms that inhabit the smallest universe \( \varType_0 \), then this relation is used
to define reducibility at level 1 for terms that live in
\( \varType_1 \) at most, and so on. Therefore, the whole definition is
done by induction on \( \ell \).
%
This construction seems deeply incompatible with an impredicative
universe, as dependent products in \( \sProp \) may mention arbitrarily
large universes in their domain while still being at the bottom
of the universe hierarchy.

Thus, replicating the usual definition for the impredicative universe
leads to a non well-founded relation.
%
In this section, we show how to break out of that circularity in the case of an
impredicative universe with definitional proof-irrelevance, by
defining a weaker logical relation on impredicative types that does
not require knowledge of the logical relation for the predicative
hierarchy.
%
However, a weaker logical relation means a weaker induction hypothesis,
which makes the fundamental lemma harder to prove.
We compensate for this by defining an alternative typing system
for \SetoidCC, called \emph{paranoid} in reference to
\sidecitet{andrej17:paranoid} and noted with $\dashPar$,
in which we add seemingly redundant hypotheses in the typing derivations.
These hypotheses are used during the proof of the fundamental lemma, to
recover information that cannot be collected in the logical relation.
%
Actually, the proof of the fundamental lemma requires to generalize
the induction hypothesis to ensure stability of reducibility, and
consider a notion called \emph{validity}, noted for equality of terms as
$\eqtmV{\ell}{\Gamma}{t}{u}{A}{\Type{}}$.
%
Validity is a generalization of reducibility, which coincides with it
when considering the identity substitution.
%
Then, once the fundamental lemma is proven, we can show that the
economic and paranoid typing systems for \SetoidCC are equivalent and
conclude normalization.
%
Finally, to get decidability of conversion for
\SetoidCC, one needs to introduce the notion of algortihmic
conversion, noted $\algoeqtmannotated{\Gamma}{t}{u}{A}{\Type{}}$,
which is sound with respect to conversion and for which we can
show that it is implied by reducible equality.
%
\cref{fig:proof-outline} summarizes the connections between those relations.

\begin{figure}
%     \[
%       \xymatrix@R=4em@C=7em{
%         \eqtmannotated{\Gamma}{t}{u}{A}{\Type{}}
%         & \eqtmParaannotated{\Gamma}{t}{u}{A}{\Type{}}
%         \ar@2{->}[l]^{\small\begin{array}{c}\mathrm{}\end{array}}_{\small\begin{array}{c}\mathrm{\cref{thm:nonparanoid}}\end{array}}
%         \ar@2{->}[d]^{}_{\small\begin{array}{c}\mathrm{Fundamental} \\
%                             \mathrm{\cref{thm:fundamental}} \end{array}}
%         &
%         \algoeqtmannotated{\Gamma}{t}{u}{A}{\Type{}}
%         \ar@2{->}[l]_{\small\begin{array}{c}\mathrm{Soundness~of}\end{array}}^{\small\begin{array}{c}\mathrm{Algo.~Conv.}\end{array}}
%         \\
%                 & \eqtmV{\ell}{\Gamma}{t}{u}{A}{\Type{}}
%                 \ar@2{->}[r]^{\small\begin{array}{c}\mathrm{Identity}\end{array}}_{\small\begin{array}{c}\mathrm{Substitution}\end{array}}
%                 &
%     \eqtmR{\ell}{\Gamma}{t}{u}{A}{\Type{}} \ar@2{->}[u]_{\small\begin{array}{c}\mathrm{Escape} \\
%                             \mathrm{Lemma} \end{array}}
%   }
%     \]
	this should be redone with tikz or smth
  \caption{Outline of the proof}
  \label{fig:proof-outline}
\end{figure}



\subsection{The Logical Relation for the Predicative Hierarchy}

We start by presenting the logical relation for the relevant
fragment, closely following \sidecitet{Abel:POPL2018,pujet:hal-03367052}.
%
For pedagogical reasons, we will present a somewhat informal version
of the logical relation where some arguments are left implicit, and
refer the interested reader to the \Agda formalization.\sidenote{The
  formalization is in the folder \texttt{logrel-ccobs}. The links to
  \Agda files are broken for the double-blind process as they point to
  our GitHub repository.}
%
While our formal proof and the definition in
\refAgda{}{LogicalRelation} only feature three levels, it should be
quite clear that the same technique works for any finite number of
levels.

The logical relation is defined by induction-recursion, as detailed
in \cref{fig:logrel-ind-rec}.
%
The logical relation on types is defined inductively with seven
constructors, one for every type of \SetoidCC. Each constructor
involves an auxiliary predicate of the form \( \Gamma \Vdash_X t \)
that packs the appropriate information for the corresponding type.
%
The logical relations on terms and equalities are simultaneously
defined by recursion on \( \tytyR{\ell}{\Gamma}{A}{s} \), using
more auxiliary predicates.
%
This technically means that these judgments should have an extra
argument of type \( \tytyR{\ell}{\Gamma}{A}{s} \), but since it
will turn out that all such derivations give rise to equivalent judgments
(cf \refAgda{}{Irrelevance}), we freely omit this argument.

\begin{figure}
  \begin{small}
\begin{flalign*}
  \begin{array}{lcl}
\_ \Vdash_\ell \_ : \_ & : & (\Gamma : \Context) \to (t : \Term) \to (A : \Term) \to (s : \Sort) \to \AgdaSet{} \\
\_ \Vdash_\ell \_ : \_ & \bnfis & \Vne : \forall\ \Gamma\ t\ s \to
                          (\Gamma \Vdash_{\mathsf{ne}} t : s) \to
                          \Gamma \Vdash_\ell t : s\\
               & \sep & \VU : \forall\ \Gamma\ t\ i \to
                          (\Gamma \Vdash_{\mathsf{U}} t : \varType_i) \to
                          \Gamma \Vdash_\ell t : \varType_i\\
               & \sep & \Vnat : \forall\ \Gamma\ t \to
                          (\Gamma \Vdash_{\Nat} t) \to
                          \Gamma \Vdash_\ell t : \varType_0\\
               & \sep & \Vpi : \forall\ \Gamma\ t\ i \to
                          (\Gamma \Vdash_{\Pi} t : \varType_i) \to
                          \Gamma \Vdash_\ell t : \varType_i\\
               & \sep & \VOmega : \forall\ \Gamma\ t\ i \to
                          (\Gamma \Vdash_{\Omega} t : \varType_i) \to
                          \Gamma \Vdash_\ell t : \varType_i\\
               & \sep & \Vforall : \forall\ \Gamma\ t \to
                          (\Gamma \Vdash_{\forall} t) \to
                          \Gamma \Vdash_\ell t : \sProp\\
               & \sep & \Vexists : \forall\ \Gamma\ t \to
                          (\Gamma \Vdash_{\exists} t) \to
                          \Gamma \Vdash_\ell t : \sProp\\
               & \sep & \Vempty : \forall\ \Gamma\ t \to
                          (\Gamma \Vdash_{\bot} t) \to
                          \Gamma \Vdash_\ell t : \sProp\\
               %% & \sep & \Vemb : \forall\ \ell'\ \Gamma\ t\ s \to
               %%            \ell' < \ell \to
               %%            \Gamma \Vdash_{\ell'} t : s \to
               %%            \Gamma \Vdash_{\ell} t : s\\
  \end{array} &&
\end{flalign*}

\begin{flalign*}
  \begin{array}{lcl}
\_ \Vdash_\ell \_ \equiv \_ : \_ & : & (\Gamma : \Context) \to (A : \Term) \to (B : \Term) \to (s : \Sort) \to \{ \Gamma \Vdash_\ell A : s \} \to \AgdaSet{} \\
\Gamma \Vdash_\ell A \equiv B : s\ \{ \VX \} & = & \Gamma \Vdash_{X,\ell} A \equiv B : s
  \qquad \text{for all }\ X \in \{\mathsf{ne}, \mathsf{U}, \mathbb{N}, \Pi, \Omega, \forall, \exists, \bot\}
  \end{array} &&
\end{flalign*}

\begin{flalign*}
  \begin{array}{lcl}
\_ \Vdash_\ell \_ : \_ : \_ & : & (\Gamma : \Context) \to (t : \Term) \to (A : \Term) \to (s : \Sort) \to \{ \Gamma \Vdash_\ell A : s \} \to \AgdaSet{} \\
\Gamma \Vdash_\ell t : A : s\ \{ \VX \} & = & \Gamma \Vdash_{X,\ell} t : A: s
  \qquad \text{for all }\ X \in \{\mathsf{ne}, \mathsf{U}, \mathbb{N}, \Pi, \Omega, \forall, \exists, \bot\}
  \end{array} &&
\end{flalign*}

\begin{flalign*}
  \begin{array}{lcl}
\_ \Vdash_\ell \_ \equiv \_ : \_ : \_ & : & (\Gamma : \Context) \to (t\ u : \Term) \to (A : \Term) \to (s : \Sort) \to \{ \Gamma \Vdash_\ell A : s \} \to \AgdaSet{} \\
\Gamma \Vdash_\ell t \equiv u : A : s\ \{ \VX \} & = & \Gamma \Vdash_{X,\ell} t \equiv u : A: s
  \qquad \text{for all }\ X \in \{\mathsf{ne}, \mathsf{U}, \mathbb{N}, \Pi, \Omega, \forall, \exists, \bot\}
  \end{array} &&
\end{flalign*}
\end{small}
  \caption{Inductive-recursive presentation of the logical relation}
  \label{fig:logrel-ind-rec}
\end{figure}

%% As is customary in reducibility proofs~\sidecite{girard72}, these
%% judgments imply normalization, are closed under weak head expansion,
%% and are verified by all neutral terms.

We now give the definition of all these auxiliary predicates, starting
with the predicates for neutral types.

\paragraph{Neutral Types}

% The first kind of reducible type are neutral type.
\[
{\small
  \inferrule{\redT{\Gamma}{A}{N}{s} \\ \neutral{N}}
  {\tytyR{\mathsf{ne}}{\Gamma}{A}s{}}
  }
\]
This rule states that $A$ is reducible to a neutral type:
\( \redT{\Gamma}{A}{N}{s} \) means that \( A \) reduces to a neutral term \( N \)
in a finite number of steps (possibly zero), and that both \( A \) and \( N \) are of sort \( s \) in context \( \Gamma \).
%
When $A$ is reducible to a neutral type, we define:
% Given a derivation
% \( \metatm{\mathscr{A}}{\tytyR{\ell}{\Gamma}{A}} \) built by this rule, we define:
\begin{itemize}
  \item \( \eqtyR{\mathsf{ne}}{\Gamma}{A}{B}{s} \) if there is a neutral term \( M \) such that
    \( \redT{\Gamma}{B}{M}{s} \) and \( \eqtm{\Gamma}{N}{M}{s} \).
  \item \( \tytmR{\mathsf{ne}}{\Gamma}{t}{A}{s} \) if there is a neutral term \( n \) such that
    \( \redT{\Gamma}{t}{n}{N} \).
  \item \( \eqtmR{\mathsf{ne}}{\Gamma}{t}{u}{A}{s} \) if there are neutral terms \( n, m \) such that
    \( \redT{\Gamma}{t}{n}{N} \)

    and \( \redT{\Gamma}{u}{m}{N} \), and
    \( \eqtm{\Gamma}{n}{m}{N} \).
\end{itemize}

Note that the definition of reducible neutral types and terms does not
recursively call the logical relation, thus it can be defined outside of
the inductive-recursive definition.

\paragraph{Natural Numbers}
\[
{\small
  \inferrule{\redT{\Gamma}{A}{\Nat}{\varType_0}}
            {\Gamma \Vdash_{\mathbb{N}} A}
}\]
When $A$ is reducible to $\Nat$, we define:
 % Given a derivation \( \metatm{\mathscr{A}}{\tytyR{\ell}{\Gamma}{A}} \) built by this rule, we
% define:
\begin{itemize}
  \item \( \Gamma \Vdash_{\mathbb{N}} A \equiv B \) if \( \redT{\Gamma}{B}{\Nat}{\varType_0} \).
  \item \( \Gamma \Vdash_{\mathbb{N}} t : A \) if there is a normal form \( t' \) such that
    \( \redT{\Gamma}{t}{t'}{\Nat} \) and \( \Gamma \Vdash_{\mathbb{N}\mathsf{t}} t' \), which is inductively
    defined by
{\small
    \[
      \inferrule{ }
                {\nattmR{\Gamma}{\zero}}
      \quad
      \inferrule{\nattmR{\Gamma}{t}}
                {\nattmR{\Gamma}{\suc{t}}}
      \quad
      \inferrule{\tytm{\Gamma}{n}{\Nat}
                \\ \text{\( n \) is neutral}}
                {\nattmR{\Gamma}{n}}
    \]}
  \item \( \Gamma \Vdash_{\mathbb{N}} t \equiv u : A \) if there are normal forms \( t', u' \) such that
    \( \redT{\Gamma}{t}{t'}{\Nat} \) and

    \( \redT{\Gamma}{u}{u'}{\Nat} \), and
    \( \nateqR{\Gamma}{t'}{u'} \), which is inductively defined by
    {\small
\[
      \inferrule{ }
                {\nateqR{\Gamma}{\zero}{\zero}}
      \quad
      \inferrule{\nateqR{\Gamma}{t}{u}}
                {\nateqR{\Gamma}{\suc{t}}{\suc{u}}}
      \quad
      \inferrule{\eqtm{\Gamma}{n}{m}{\Nat}
                \\ \text{\( n, m \) are neutral}}
                {\nateqR{\Gamma}{n}{m}}
    \]}
\end{itemize}

Again, those definitions do not
mention the logical relation and can be defined \emph{a priori}.


\paragraph{Predicative Universes}
\[
{\small
  \inferrule{\redT{\Gamma}{A}{\varType_i{}}{\varType_{i+1}}}
            {\tytyR{\mathsf{U}}{\Gamma}{A}{\varType_{i+1}}}
            { \substack{i < \ell}}
}\]
When $A$ is reducible to a predicative universe, we define:
\begin{itemize}
  \item \( \eqtyR{\mathsf{U}}{\Gamma}{A}{B}{\varType_{i+1}} \) if \( \redT{\Gamma}{B}{\varType_i{}}{\varType_{i+1}} \).
  \item \( \tytmR{\mathsf{U}}{\Gamma}{t}{A}{\varType_{i+1}} \) if there is a normal form \( t' \) such that
    \( \redT{\Gamma}{t}{t'}{\varType_i{}} \), and \( \tytyR{i}{\Gamma}{t}{\varType_i} \)

    (which is already defined by induction hypothesis, since
    \( i < \ell \)).
  \item \( \eqtmR{\mathsf{U}}{\Gamma}{t}{u}{A}{\varType_{i+1}} \) if there are normal forms \( t', u' \) such that
    \( \redT{\Gamma}{t}{t'}{\varType_i{}} \)

    and \( \redT{\Gamma}{u}{u'}{\varType_i{}} \), and
    \( \tytyR{i}{\Gamma}{t}{\varType_i} \), \( \tytyR{i}{\Gamma}{u}{\varType_i} \), and \( \eqtyR{i}{\Gamma}{t}{u}{\varType_i} \).
\end{itemize}



\paragraph{Dependent Function Types}

  \[
{\small
  \inferrule{\redT{\Gamma}{A}{\Depfunannotated{F}{G}{}{s}{\varType_i}}{\varType_j} \\ \piRel{\Type{}}{\varType_i} = \varType_j
            \\ \tytm{\Gamma}{F}{\Type{}}
            \\ \tytm{\extctx{\Gamma}{F}}{G}{\varType_i}
            \\ \forall \wkrho \to \tytyR{\ell}{\Delta}{F[\rho]}{\Type{}}
            \\ \forall \wkrho \to \metaimply{\tytmR{\ell}{\Delta}{a}{F[\rho]}{\Type{}}}{\tytyR{\ell}{\Delta}{G[\rho, a]}{\varType_i}}
            \\ \forall \wkrho \to
              \metaimply{\tytmR{\ell}{\Delta}{a}{F[\rho]}{\Type{}}}
              {\metaimply{\tytmR{\ell}{\Delta}{b}{F[\rho]}{\Type{}}}
              {\metaimply{\eqtmR{\ell}{\Delta}{a}{b}{F[\rho]}{\Type{}}}
              {\eqtyR{\ell}{\Delta}{G[\rho, a]}{G[\rho, b]}{\varType_i}}}}}
            {\tytyR{\Pi}{\Gamma}{A}{\varType_j}}
          }\]
%
This rule states that $A$ is reducible to a dependent function type.
%
We introduced quite a bit of notation here. \( \to \) is implication in the meta
theory, \ie the theory of \Agda. \( \forall \wkrho \) is a meta-theoretical
quantification on an arbitrary well-formed context \( \Delta \) and a weakening \( \rho \)
that turns \( \Delta \) into \( \Gamma \).
Applying such a weakening on the free variables of a term \( F \) that is a well-typed
in context \( \Gamma \) results in a term \( F[\rho] \) that is well-typed in context \( \Delta \).
Given a term \( G \) in the extended context \( \extctx{\Gamma}{F} \) and a term
\( \tytm{\Delta}{a}{F[\rho]} \), we can apply \( \rho \) on the free variables of \( G \) except
for \( x \) to get a term in \( \extctx{\Delta}{F[\rho]} \), and then substitute \( x \) with \( a \)
to get a term in \( \Delta \). The result is noted \( G[\rho, a] \).

With those notations, a term $A$ is reducible to a dependent function
type when (i) it reduces to $\Depfun{F}{G}$, (ii) $F$ is a reducible type, (iii)
given any reducible term $a$ at type $F$, $G[a]$ is a reducible type and (iv) when $a$ and $b$ are reducibly equal at type $B$, $G[a]$
and $G[b]$ are reducibly equal types.
%
Moreover, all these reducibility premises should hold under any weakening \( \rho \).
%
Note that the definition above defines reducibility of a type using
reducibility of terms, which explains the need for an inductive-recursive definition.

When $A$ is reducible to a dependent function type,
% Given a derivation \( \metatm{\mathscr{A}}{\tytyR{\ell}{\Gamma}{A}}
% \) built by this rule,
we define:
\begin{itemize}
  \item \( \eqtyR{\Pi}{\Gamma}{A}{B}{\varType_j} \) if there are terms \( F' \) and \( G' \) such that
    \begin{itemize}
      \item \( \redT{\Gamma}{B}{\Depfun{F'}{G'}}{\varType_j} \) and
        \( \eqtm{\Gamma}{\Depfun{F}{G}}{\Depfun{F'}{G'}}{\varType_j} \)
      \item \( \forall \wkrho.\ \ \eqtyR{\ell}{\Delta}{F[\rho]}{F'[\rho]}{\Type{}} \)
      \item \( \forall \wkrho.\ \ \metaimply{\tytmR{\ell}{\Delta}{a}{F[\rho]}{\Type{}}}{\eqtyR{\ell}{\Delta}{G[\rho, a]}{G'[\rho, a]}{\varType_i}} \).
      \end{itemize}
    \item \( \tytmR{\Pi}{\Gamma}{t}{A}{{\varType_j}} \) if there is a normal form \( t' \) such that
    \begin{itemize}
      \item \( \redT{\Gamma}{t}{t'}{\Depfun{F}{G}} \)
      \item \( \forall \wkrho.\ \
        \metaimply{\tytmR{\ell}{\Delta}{a}{F[\rho]}{\Type{}}}{\tytmR{\ell}{\Delta}{t'[\rho]\ a}{G[\rho, a]}{\varType_i}} \)
      \item
        $\forall \wkrho.  \  \
              \metaimply{\tytmR{\ell}{\Delta}{a}{F[\rho]}{\Type{}}}
                    {\metaimply{\tytmR{\ell}{\Delta}{b}{F[\rho]}{\Type{}}}
                      {\eqtmR{\ell}{\Delta}{a}{b}{F[\rho]}{\Type{}}}}$

                          \hspace{9.1em}
                          $\metaimply{}{\eqtmR{\ell}{\Delta}{t'[\rho]\ a}{t'[\rho]\ b}{G[\rho,
                                                                    a]}{\varType_i}}$.
    \end{itemize}
  \item \( \eqtmR{\Pi}{\Gamma}{t}{u}{A}{\varType_j} \) if there are normal forms \( t', u' \) such that
    \begin{itemize}
      \item \( \redT{\Gamma}{t}{t'}{\Depfun{F}{G}} \)
      and \( \redT{\Gamma}{u}{u'}{\Depfun{F}{G}} \)
      \item \( \eqtm{\Gamma}{t'}{u'}{\Depfun{F}{G}} \)
      \item \( \tytmR{\ell}{\Gamma}{t}{A}{\varType_j} \) and \( \tytmR{\ell}{\Gamma}{u}{A}{\varType_j} \)
      \item \( \forall \wkrho.\ \ \metaimply{\tytmR{\ell}{\Delta}{a}{F[\rho]}{\Type{}}}{\eqtmR{\ell}{\Delta}{t'[\rho]\ a}{u'[\rho]\ a}{G[\rho, a]}{\varType_i}} \).
    \end{itemize}
  \end{itemize}

  The three definitions above make sense because we know that when
  $A$ is reducible to a dependent function type, we also know that its domain $F$
  and codomain $G$ are reducible.
  %% The reducibility of a term at a  dependent function type is defined pointwise, but we also need
  %% to ask that the term preserves reducible equality.

\subsection{The Logical Relation for the Impredicative Universe}

We now turn to the definition of the relation for the proof-irrelevant
fragment, which must be defined independently from the logical relation
to avoid a non well-founded definition.

Since there is no computation whatsoever in proof-irrelevant types,
their inhabitants don't interact with other terms. This allows us
to give a generic definition for reducibility of terms and term
equality, that will work for any \( X \in \{ \bot, \forall, \exists\} \) :
%
\begin{itemize}
  \item \( \tytmR{X}{\Gamma}{t}{A}{\sProp}\ \) when \(\ \tytm{\Gamma}{t}{A} \).
  \item \( \eqtmR{X}{\Gamma}{t}{u}{A}{\sProp}\ \) when \(\ \tytm{\Gamma}{t}{A} \) and \( \tytm{\Gamma}{u}{A} \).
  \end{itemize}
%
which means that if \( A \) is in \( \sProp \), then the logical
relation does not need to collect any information on inhabitants of
\( A \), save for the fact that they are well-typed. And this applies
to equality too, for any two inhabitants of \( A \) are always
convertible.

It only remains to define the reducibility of types and type equality
for $\Empty$, impredicative dependent products and existential types.
%
We do not add a case for \( \Unit \), since it can be encoded
as \( \Fun{\Empty}{\Empty} \).
%
Note that the observational equality is not featured in the the logical
relation, as it does not play the role of a type constructor, but
rather that of a detructor that computes by pattern-matching on types.

\paragraph{Empty type}

The case of the empty type is easy, as it does not recursively call
the logical relation.
\[
{\small
  \inferrule{\redT{\Gamma}{A}{\Empty}{\sProp[i]}}
            {\Gamma \Vdash_{\bot} A}
          }\]
%
        When $A$ is reducible to the empty type, we define
        \( \Gamma \Vdash_{\bot} A \equiv B \) as \( \redT{\Gamma}{B}{\Empty}{\sProp[i]} \).


\paragraph{Impredicative Dependent Function Types}

For the impredicative function types, the situation is more complex,
as we can not reproduce the definition of their predicative
counterpart:
it involves a recursive call to the logical relation for the domain
and codomain types, which might live in a higher universe than the
function type.
%
Consequently, we go for minimalism and only collect the fact that the domain
and the codomain are well-typed.

  \[
{\small
  \inferrule{\redT{\Gamma}{A}{\Depfunannotated{F}{G}{}{s}{\sProp}}{\sProp}
            \\ \tytm{\Gamma}{F}{\Type{}}
            \\ \tytm{\extctx{\Gamma}{F}}{G}{\sProp}}
            {\Gamma \Vdash_\forall A}
          }\]
%
Similarly, when $A$ is reducible to an impredicative function type,
% Given a derivation \( \metatm{\mathscr{A}}{\tytyR{\ell}{\Gamma}{A}}
% \) built by this rule,
we define reducible equality of $A$ and $B$ as the fact that $B$
reduces to a convertible impredicative function type:
  \[
{\small
  \inferrule{ \redT{\Gamma}{B}{\Depfun{F'}{G'}}{\sProp}
            \\  \eqtm{\Gamma}{\Depfun{F}{G}}{\Depfun{F'}{G'}}{\sProp}}
            {\Gamma \Vdash_\forall A \equiv B}
          }\]
        %

These definitions do not recursively call the logical relation,
but as a result the collected invariants are
much weaker than those for relevant dependent function
types. We will see in \cref{sec:fundamental-lemma} how to
circumvent this problem by using a paranoid type system with
many redundant hypotheses.


\paragraph{Existential types}

The definition of the logical relation for existential types is
similar to the one for impredicative dependent function types. We define
  \[
{\small
  \inferrule{\redT{\Gamma}{A}{\Exists{F}{G}}{\sProp}
            \\ \tytm{\Gamma}{F}{\sProp}
            \\ \tytm{\extctx{\Gamma}{F}}{G}{\sProp}}
            {\Gamma \Vdash_\exists A}
          }\]
%
  \[
{\small
  \inferrule{ \redT{\Gamma}{B}{\Exists{F'}{G'}}{\sProp}
            \\  \eqtm{\Gamma}{\Exists{F}{G}}{\Exists{F'}{G'}}{\sProp}}
            {\Gamma \Vdash_\exists A \equiv B}
          }\]
          %
The only remaining case of our definition is the impredicative universe $\sProp$.

\paragraph{The Impredicative Universe}
\[
{\small
  \inferrule{\redT{\Gamma}{A}{\sProp}{\varType_0}}
            {\tytyR{\Omega}{\Gamma}{A}{\varType_0}}
}\]
When $A$ is reducible to the impredicative universe, we define:
\begin{itemize}
  \item \( \eqtyR{\Omega}{\Gamma}{A}{B}{\varType_i} \) if \( \redT{\Gamma}{B}{\sProp}{\varType_i} \).
  \item \( \tytmR{\Omega}{\Gamma}{t}{A}{\varType_j} \) if there is a normal form \( t' \) such that
    \( \redT{\Gamma}{t}{t'}{\sProp} \), and \( \tytyR{i}{\Gamma}{t}{\sProp} \)

    (which is already defined as reducibility of impredicative types is defined beforehand).
  \item \( \eqtmR{\Omega}{\Gamma}{t}{u}{A}{\varType_j} \) if there are normal forms \( t', u' \) such that
    \( \redT{\Gamma}{t}{t'}{\sProp} \)

    and \( \redT{\Gamma}{u}{u'}{\sProp} \), and
    \( \tytyR{i}{\Gamma}{t}{\sProp} \), \( \tytyR{i}{\Gamma}{u}{\sProp} \), and \( \eqtyR{i}{\Gamma}{t}{u}{\sProp} \).
\end{itemize}

\begin{figure}
  \begin{small}
    \begin{mathpar}
 \inferrule[$\Pi$-Form$_p$] {\tytmPara{\Gamma}{A}{\Type[]{}} \\
            \tytmPara{\extctxannotated{\Gamma}{A}{\Type[]{}}}{B}{\Type[]{'}}}
            {\tytmPara{\Gamma}{\Depfunannotated{A}{B}{s}{s}{s'}}{\piRel{\Type}{\Type[]{'}}}}
            {}
\ilabel{infrule:pi-form-Para}
\and
 \inferrule[Fun$_p$]{\tytmPara{\Gamma}{A}{\Type[]{}} \\
   \tytmParaannotated{\extctxannotated{\Gamma}{A}{\Type{}}}{t}{B}{\Type{'}}}
            {\tytmParaannotated{\Gamma}{\lam{A}{t}}{\Depfun{A}{B}}{\piRel{\Type}{\Type[]{'}}}}
\ilabel{infrule:pi-intro-Para}
  \\
  \inferrule[App$_p$]{\tytmPara{\Gamma}{A}{\Type[]{}} \\
    \tytmPara{\extctxannotated{\Gamma}{A}{\Type[]{}}}{B}{\Type[]{'}}
    \\
    \tytmParaannotated{\Gamma}{t}{\Depfun{A}{B}}{\piRel{\Type}{\Type[]{'}}}
            \\ \tytmParaannotated{\Gamma}{u}{A}{\Type{}}}
            {\tytmParaannotated{\Gamma}{t\ u}{\subst{B}{u}}{\Type[]{'}}}
            \ilabel{infrule:pi-elim-Para}
  \\
  \inferrule[$\exists$-Form$_p$]{\tytmPara{\Gamma}{A}{\sProp[i]} \\
             \tytmPara{\extctxannotated{\Gamma}{A}{\sProp}}{B}{\sProp[i]}}
            {\tytmPara{\Gamma}{\Exists{A}{B}}{\sProp[i]}}
            % {\scriptstyle \substack{i \le k \\ j \le k}}
            \ilabel{infrule:exists-form-Para}
            \and
              \inferrule[Fst$_p$]{    \tytmPara{\Gamma}{A}{\sProp[i]} \\
     \tytmPara{\extctxannotated{\Gamma}{A}{\sProp}}{B}{\sProp[i]} \\\tytmParaannotated{\Gamma}{t}{\Exists{A}{B}}{\sProp}}
            {\tytmParaannotated{\Gamma}{\fst{t}}{A}{\sProp}}\ilabel{infrule:fst-Para}
\\
  \inferrule[Snd$_p$]{    \tytmPara{\Gamma}{A}{\sProp[i]} \\
     \tytmPara{\extctxannotated{\Gamma}{A}{\sProp}}{B}{\sProp[i]} \\\tytmParaannotated{\Gamma}{t}{\Exists{A}{B}}{\sProp}}
  {\tytmParaannotated{\Gamma}{\snd{t}}{\subst{B}{\fst{t}}}{\sProp}}
  \ilabel{infrule:snd-Para}
   \\
  \inferrule[Pair$_p$]{
    \tytmPara{\Gamma}{A}{\sProp[i]} \\
     \tytmPara{\extctxannotated{\Gamma}{A}{\sProp}}{B}{\sProp[i]} \\ \tytmParaannotated{\Gamma}{t}{A}{\sProp}
            \\ \tytmParaannotated{\Gamma}{u}{\subst{B}{t}}{\sProp}}
          {\tytmParaannotated{\Gamma}{\pair{t}{u}}{\Exists{A}{B}}{\sProp}}
          \ilabel{infrule:pair-Para}
        \end{mathpar}
        \end{small}
        \caption{\SetoidCC Paranoid Typing Rules (dependent function
          types and existential types only)}
  \label{fig:SetoidCC-paranoid-typing}
\end{figure}


\subsection{A Paranoid Version of \SetoidCC }

In order to avoid circularity, we weakened the definition of the
logical relation for dependent products, so that it does not mention
reducibility of the domain and the codomain.
%
In counterpart, we need to prove the fundamental lemma with a
significantly weaker induction hypothesis for proof-irrelevant type
constructors.
%
To remedy to this problem, we define a paranoid variant of the type
system of \SetoidCC that includes seemingly redundant premises in the
typing rules of dependent products and existential types.
%
This variant, noted $\dashPar$, is presented in \cref{fig:SetoidCC-paranoid-typing}. The idea is that every
sub-term of the conclusion must be explicitly typed
in the premises.
%
For instance, the rule for application not only asks for $t$ and $u$ to
be well-typed, but also for domain $A$ and codomain $B$ of the dependent
function type of $t$.

Obviously, when a term is well-typed for the paranoid variant, it is
also well-typed in the economic system. But after proving the
fundamental lemma on the paranoid variant, it becomes possible to show
that the two typing systems are equivalent, leveraging the inversion
lemmas provided by the logical relation.
%
Thus, the additional premises of the paranoid variant help us
prove the fundamental lemma in presence of impredicativity, but end up
being redundant once a sufficient amount of metatheory is established.

\subsection{The Fundamental Lemma}
\label{sec:fundamental-lemma}

The fundamental lemma expresses the completeness of the logical
relation with respect to
typing~\refAgda{LogicalRelation.}{Fundamental}.
%
But before starting our induction, we need to generalize our
hypothesis---the logical relation---yet a bit more: we will prove
reducibility under any well-formed substitution.
%
Following \sidecite{Abel:POPL2018}, we use induction-recursion to define
the notion of \emph{validity}, which will serve as our actual induction
hypothesis \refAgda{LogicalRelation.}{Substitution}.
%

\[ {\small \inferrule{\forall \Delta\ \sigma \to \
      \metaimply{\ctxV{\Delta}}{\metaimply{\tytyS{\Delta}{\sigma}{\Gamma}}{
          \left \{ \begin{array}{l}
                     \tytyR{\ell}{\Delta}{A[\sigma]}{\Type{}} \\
                     \forall \sigma' \to
                     \metaimply{\tytyS{\Delta}{\sigma'}{\Gamma}
                     }{\metaimply{\eqtyS{\Delta}{\sigma}{\sigma'}{\Gamma}}{\eqtyR{\ell}{\Delta}{A[\sigma]}{A[\sigma']}{\Type{}}
                     }}
               \end{array}\right . }}
}
{\tytyV{\ell}{\Gamma}{A}{\Type{}}}
}\]
%
In the definition above, $\Delta$ is valid context (noted $\ctxV{\Delta}$, meaning
that the context is composed of valid types) and a
substitution $\sigma$ is valid, noted $\tytyS{\Delta}{\sigma}{\Gamma}$,
when it substitutes only reducible terms. Similarly, two valid
substitutions $\sigma$ and $\sigma'$ are equal, noted
$\eqtyS{\Delta}{\sigma}{\sigma'}{\Gamma}$, when they substitute
reducibly equal terms.

There are similar notions of validity for terms, equality of
types and equality of terms. For instance, when $A$ is a valid
type, the validity of a term $t$ at $A$ is defined as

\[
{\small
  \inferrule{\forall \Delta\ \sigma \to \metaimply{\ctxV{\Delta}}{\metaimply{\tytyS{\Delta}{\sigma}{\Gamma}}{
        \left \{ \begin{array}{l}
           \tytmR{\ell}{\Delta}{t[\sigma]}{A[\sigma]}{\Type{}} \\
                   \forall \sigma' \to
                   \ \metaimply{\tytyS{\Delta}{\sigma'}{\Gamma} }{\metaimply{\eqtyS{\Delta}{\sigma}{\sigma'}{\Gamma}}{\eqtmR{\ell}{\Delta}{t[\sigma]}{t[\sigma']}{A[\sigma]}{\Type{}} }}
                 \end{array} \right .}}
         }
         {\tytmV{\ell}{\Gamma}{t}{A}{\Type{}}}
       }
\]

%% This notion of validity is a way to enforce stability by substitution
%% for reducible terms.
%
Then, the fundamental lemma is stated with the paranoid variant of the type
system. The lemma is proven using a mutual induction on derivations for well-typed types, terms, equality of
types and equality of terms. We only present the statement on
well-typed terms as we only explain the difference from the proof of
\sidecitet{Abel:POPL2018,pujet:hal-03367052} on the impredicative fragment.
%
Note that because validity of a term can only be defined if the
context and the type are themselves valid, the fundamental lemma
actually provides validity of every component of the typing judgment.
%
\begin{lemma}[Fundamental lemma (for terms)
  \refAgda{LogicalRelation.}{Fundamental}]
  \label{thm:fundamental}
  \
  If\, \( \tytmParaannotated{\Gamma}{t}{A}{\Type{}} \), then there is a level \( \ell \) such that $\ctxV{\Gamma}$,
  \( \tytyV{\ell}{\Gamma}{A}{\Type{}} \) and \( \tytmV{\ell}{\Gamma}{t}{A}{\Type{}} \).
\end{lemma}
\begin{proof}
  The theorem is proven by induction on the paranoid derivation.
  %
  The only cases where the proof differs from
  \sidecitet{Abel:POPL2018,pujet:hal-03367052} is when a rule builds
  an inhabitant of a proof-irrelevant type. In this case, we need
  to show that (i) the proof-irrelevant type is valid, and
  (ii) that the term is well-typed.
  %
  Showing that the term is well-typed is always straightforward, but
  the validity of its type can be more difficult to obtain.

  In the case of the application of functions (Rule \nameref{infrule:pi-elim-Para}), the usual
  predicative way to get the fact that ${\subst{B}{u}}$ is valid
  is to extract it from the information collected by the
  validity of the dependent function type (which is known
  by induction hypothesis).
  %
  In the impredicative case however, we do not get such expansive information
  from the validity of the dependent function type, because validity
  of a type in the impredicative universe cannot recursively mention validity of the codomain, just like reducibility.
  %
  This is precisely where the paranoid assumption comes into play:
  %
  from the induction hypothesis on the additional assumptions, we also
  know that $B$ is valid as a type family over $A$. Since validity is
  defined as reducibility under any valid substitution, we can get the
  reducibility of ${\subst{B}{u}}$.
  %

  This technique also applies to other rules such as \nameref{infrule:fst-Para}
  or \nameref{infrule:snd-Para}.
  %
\end{proof}

\subsection{Removing Paranoid Premises}
\label{sec:remov-paran-prem}

So far, we established the fundamental lemma on the paranoid
version of the type system, but we would like to work with the more
economic presentation instead.

The first step is to establish the following corollary on the paranoid
typing, by combining the fundamental lemma with the fact that
reducible types are well-formed and reducible terms are well-typed:
%
\begin{corollary}[Typing Validity \refAgda{Typed.Consequences.}{Syntactic}] \label{thm:validity}
  If $\tytmParaannotated{\Gamma}{t}{A}{\Type{}}$ then $\tytmPara{\Gamma}{A}{\Type{}}$ and similarly, if $\eqtmParaannotated{\Gamma}{t}{u}{A}{\Type{}}$ then
  $\tytmPara{\Gamma}{A}{\Type{}}$,
  $\tytmParaannotated{\Gamma}{t}{A}{\Type{}}$ and $\tytmParaannotated{\Gamma}{u}{A}{\Type{}}$.
\end{corollary}
%
And now that we are equipped with this corollary, as well as some
inversion lemmas that we obtained from the logical relation, we know
enough to build a paranoid typing derivation from a
standard derivation:
%
\begin{theorem}[\refAgda{Typed.}{NonParanoidTyping}]
  \label{thm:nonparanoid}
  If $\tytmannotated{\Gamma}{t}{A}{\Type{}}$ then $\tytmParaannotated{\Gamma}{t}{A}{\Type{}}$.
\end{theorem}
\begin{proof}
  The proof is by induction on the typing derivation. The
  correspondence between the two systems is one-to-one except for the
  additional paranoid assumptions.
  %
  In all cases, we can get them from \cref{thm:validity} and from
  inversion lemmas, such as the fact that $A$ and $B$ are well-typed
  whenever $\Depfun{A}{B}$ is.
\end{proof}

Thus, we also have the fundamental lemma for the economic type system:
%
\begin{corollary}[Fundamental lemma on the economic type system]

  \

  If\, \( \tytmannotated{\Gamma}{t}{A}{\Type{}} \), then there is a level \( \ell \) such that $\ctxV{\Gamma}$,
  \( \tytyV{\ell}{\Gamma}{A}{\Type{}} \) and \( \tytmV{\ell}{\Gamma}{t}{A}{\Type{}} \).
\end{corollary}


A direct consequence of the fundamental lemma is that any well-typed
term has a weak-head normal form.
%
Another consequence is that any closed
term of type $\Nat$ reduces to a whnf of type $\Nat$. Thus, to
obtain canonicity for the integers of \SetoidCC, we just need to
know that there is no neutral term of type $\Nat$ in an empty context
\refAgda{Typed.Consequences.}{Canonicity}.

Unfortunately, as in \sidecite{pujet:hal-03367052}, the logical relation
is not sufficient to establish this. In fact, as we will see in \cref{sec:analys-norm-proof}, there is a good reason for that: the
consistency and canonicity theorems for \SetoidCC have a remarkably high
proof-theoretical strength due to impredicativity, whereas the
logical relation argument can be developed in a predicative
meta-theory.
%
In \cref{sec:cons-seto-model}, we will supplement
 the normalization result with
a model of \SetoidCC in an impredicative meta-theory, from which
we will deduce the consistency and canoncity theorems.

\subsection{Decidability of Conversion}

Besides proving weak-head normalization of well-typed terms, the
main point of the fundamental lemma is to be able to show that
conversion is decidable in $\SetoidCC$.
%
To do so, we define an algorithmic relation
$\algoeqtmannotated{\Gamma}{t}{u}{A}{\Type{}}$ which is supposed to
simulate convertibility of two terms $t$ and $u$, while being
easier to decide \refAgda{}{Conversion}.
%
This algorithmic conversion keeps track of the relevance information,
and uses it to either give an immediate answer in case \( t \) and
\( u \) are irrelevant, or compute their weak-head normal forms, then
compare their head constructor, and possibly apply the algorithm
recursively in case \( t \) and \( u \) are relevant.
%
Correctness of this algorithmic conversion is direct as the rules used are
subsumed by the general conversion judgement \refAgda{Conversion.}{Soundness}.

Showing that the algorithmic conversion is also complete is more complex.
%
It basically amounts to replaying the fundamental lemma with a definition of the logical
relation that uses algorithmic conversion instead of the general conversion
\refAgda{Conversion.Consequences.}{Completeness}.
%
In our formal proof, we follow \sidecitet{Abel:POPL2018} in factoring the two
instances of the fundamental lemma by defininig a generic interface for both
algorithmic conversion and typed conversion, and using this interface
in the definition of the logical relation \refAgda{Typed.}{EqualityRelation}.

\begin{theorem}[Equivalence of conversion and algorithmic conversion]
  \label{thm:algoconv}
    Given two terms $t$ and $u$ such that
    $\tytmannotated{\Gamma}{t}{A}{\Type{}}$ and
    $\tytmannotated{\Gamma}{u}{A}{\Type{}}$, we have that

    $$ \eqtmannotated{\Gamma}{t}{u}{A}{\Type{}} \Longleftrightarrow\algoeqtmannotated{\Gamma}{t}{u}{A}{\Type{}}.$$
\end{theorem}

It still remains to provide a decision procedure for the algorithmic
conversion \refAgda{Conversion.}{Decidable}.
%
Given two terms $t$ and $u$ well-typed at $A : \Type{}$ in context
$\Gamma$, we can apply the fundamental lemma plus the reflexivity of
the logical relation to get the fact that $\eqtmR{\ell}{\Gamma}{t}{t}{A}{\Type{}}$
and similarly for $u$.
%
Because reducibly equal terms are also algorithmically convertible, we
have $\algoeqtmannotated{\Gamma}{t}{t}{A}{\Type{}}$ (and similarly for $u$).
%
Then the decision procedure is done by double induction on the proofs
that $t$ and $u$ are reflexive for the algorithmic conversion.
%
The idea is that $t$ and $u$ are convertible if and only if the two reflexive proofs are the same.

Note that the proof that $t$ is algorithmically equal to itself
contains the fact that $t$ strongly normalizes, because
$t$ can be recursively put in whnf.

\section{Analysis of the Normalization Proof}
\label{sec:analys-norm-proof}

The informal proof presented in \cref{sec:logic-relat-with}, and
formalized in \Agda, outlines the construction of a normalization
model in a rather powerful metatheory: Martin-Löf Type Theory extended
with induction-recursion.
%
However, in order to control the computational complexity of the proof terms
more finely, we can try to weaken this metatheory as much as possible.

\subsection{The Computational Power of \SetoidCC}

To establish a simple lower bound, we can start by noting that Martin-Löf
Type Theory is a subset of \SetoidCC. Indeed, dependent products and
the universes are handled in the exact same way, and \sidecitet{pujet:hal-03367052} explain
how to deal with inductive tyes.
%
Therefore, a normalization proof for \SetoidCC can also act as a
normalization proof for \MLTT.
%
And since normalization entails consistency for \MLTT, it follows that our
meta-theory needs to be at least as strong as \MLTT---in fact, it turns out to be exactly what we need.

\begin{theorem}\label{setoidCC-in-MLTT}
  \MLTT with \( n + 4 \) universes can prove normalization of \SetoidCC with
  \( n \) universes.
\end{theorem}

We defer the proof of this theorem to \cref{sec:fitt-norm-proof} and
first analyse what can be deduced from it.

\begin{corollary}
  Normalization for \SetoidCC is equivalent to normalization for \MLTT
  over a weak fragment of arithmetic.
\end{corollary}

Keep in mind that we cannot hope for a similar result for consistency or
canonicity: while consistency and canonicity for \MLTT follow from
normalization, proving that \SetoidCC is consistent requires the increased
power of impredicativity.

\cref{setoidCC-in-MLTT} also provides us with some
control over the integer functions of \SetoidCC:

\begin{corollary}\label{integer-functions}
  \SetoidCC and \MLTT can express the exact same integer functions
  as closed terms of type \( \Fun{\Nat}{\Nat} \).
\end{corollary}

\begin{proof}
  As already explained, functions from \MLTT can simply be embedded in
  \SetoidCC.
  %
  For the converse direction, note that any closed term \( f \) of
  type \( \Fun{\Nat}{\Nat} \) in \SetoidCC only mentions a finite number of
  universes.
%
  Thus, \cref{setoidCC-in-MLTT} provides us with a
  normalization function for the corresponding fragment of \SetoidCC,
  that computes a normal form when applied to \( f n \) where \( n \)
  is a closed syntactic integer.
%
  If we compose it with a function that sends normal forms to the
  adequate integer, %(or 0 if the normal form is non-canonical)
  we obtain an integer function in \( \MLTT \), and canonicity ensures
  that it represents the same function as the original \SetoidCC term.
\end{proof}

%
\cref{integer-functions} might come off as a surprise, since \SetoidCC is equipped with an impredicative universe,
and impredicativity generally adds a great deal of proof-theoretical
strength!
%
And \SetoidCC does possess this logical power, in fact. Indeed, it can
represent many more functions as \( \sProp \)-valued functional relations
than \MLTT. But this corollary shows that there is no way to extract them
in the proof-relevant fragment, even though we have access to elimination
principles for false propositions and equalities (using casts).
%
Thus, the logical power of impredicativity is locked inside of \( \sProp \).
%
This is in stark contrast with the Calculus of Inductive Constructions,
in which it is possible to define closed terms of type \( \Fun{\Nat}{\Nat} \) that leverage
the power of impredicativity, using a principle called large elimination
of singleton inductive types\footnote{The standard technique is to use
the accessibility predicate as defined in \url{https://coq.inria.fr/library/Coq.Init.Wf.html}.}.
%
We discuss this principle with more detail in \cref{sec:setoidcc-with-an}.

\subsection{Fitting the Normalization Proof in MLTT}
\label{sec:fitt-norm-proof}

In this subsection, we give a proof of \cref{setoidCC-in-MLTT}, by
encoding the logical relation sketched in \cref{sec:logic-relat-with}
without induction-recursion.
%
This argument has been formalized in \Agda.\footnote{The formalization is in the
  folder \texttt{logrel-wo-ind-rec}. It is done for \MLTT and not
full \SetoidCC, but it does not make a difference as the crux is in
the definition of the relation without induction-recursion.}
%
For the remainder of this section, we work in plain \MLTT with a deep
embedding of \SetoidCC.
%
To avoid confusion between the meta-theory and the object theory, we
will use \Agda-style notations for the meta-theory.

From a bird's eye perspective, the normalization proof builds a model
of \SetoidCC in which well-formed types are interpreted as proof-relevant
predicates on untyped terms, and induction-recursion is used to construct a
universe in the model: we define the inductive predicate of reducible types
simultaneously with recursive functions that associates reducibility
predicates to a reducible type.
%
On closer inspection, we realize that the proof still works if the
reducibility predicate of a universe lives one universe higher than the
reducibility predicates of the types it contains. This allows us to use
\emph{small} induction-recursion, which can be replaced by functional
relations that only require simple indexed inductive types
\sidecite{small-ind-rec}.

\cref{fig:logrel-ind} showcases what the relation-based definition
looks like.
%
For all \( \ell \le n \), we define an inductive relation \( \AgdaData{R}^\ell \) between
a context, an untyped term \( t \) (the reducible type), its sort, a predicate \( P_{=} \) of
types that are convertible to \( t \), a predicate \( P_{t} \) which is the
reducibility predicate associated to \( t \), and a binary relation
\( P_{t=} \) that encodes convertibility of terms in \( P_{t} \).
%
The logical relation \( \Vdash_\ell \) is then defined in terms of \( \AgdaData{R}^\ell \).
%
The inductive relation features eight constructors that are each
defined in \cref{sec:logic-relat-with}, including \( \Rpi \)
which recursively calls the logical relation.
%% , plus an eighth one that ensures
%% \( \AgdaData{R}^\ell \) contains all the relations \( \AgdaData{R}^{\ell'} \) for \( \ell' < \ell \).

Note that the logical relation is built in stages:
\begin{itemize}
\item We first define an inductive relation
  \( \AgdaData{R}^0 \) that has cases for dependent products and all base types, except
  for proof-relevant universes: it only accounts for inhabitants of
  \( \varType_0 \).
  %
\item From \( \AgdaData{R}^0 \), we define a reducibility predicate \( P_{\varType_0} \) for
  \( \varType_0 \): a term \( t \) is in \( P_{\varType_0} \) if there exist
  \( P_=, P_t, P_{t=} \) such that \( R^0(t,P_=, P_t, P_{t=}) \).
  %
  Now that we have a predicate for \( \varType_0 \), we can define a relation
  \( \AgdaData{R}^1 \) that accounts for inhabitants of \( \varType_1 \). Of course, since
  \( P_{\varType_0} \) lives in \( \AgdaSet{1} \), \( \AgdaData{R}^1 \) will land in \( \AgdaSet{2} \).
\item We repeat this process \( n \) times to obtain a relation \( \AgdaData{R}^n \)
  that handles our \( n \) universes.
\end{itemize}
Each step of this process requires an additional meta-theoretical universe
level.
%
This is only natural, since we are proving normalization for a theory that
subsumes \( \MLTT_n \), a property from which we can deduce the consistency
of \( \MLTT_n \).
%
Therefore, Gödel's incompleteness theorem applies and guarantees that
we need more universes in the meta-theory than in the fragment we
consider.


\begin{figure}
  \begin{small}
\begin{flalign*}
  \begin{array}{lcl}
   \AgdaData{R}^\ell         & : & \Context \to \Term \to (\Term \to \AgdaSet{\ell}) \to (\Term \to \AgdaSet{\ell}) \to (\Term \to \Term \to \AgdaSet{\ell}) \to \AgdaSet{\ell+1} \\
   \AgdaData{R}^\ell         & \bnfis & \Rne : \forall\ \Gamma\ t \to
                          (\Gamma \Vdash_{\mathsf{ne}} t) \to
                          \AgdaData{R}^\ell\ \Gamma\ t\ (\Gamma\ \Vdash_{\mathsf{ne}} t \equiv \_)\ (\Gamma\ \Vdash_{\mathsf{ne}} \_ : t)\ (\Gamma\ \Vdash_{\mathsf{ne}} \_ \equiv \_ : t)\\
               & \sep & \RU : \forall\ \Gamma\ t \to
                          (\Gamma \Vdash_{\mathsf{U}} t) \to
                          \AgdaData{R}^\ell\ \Gamma\ t\ (\Gamma\ \Vdash_{\mathsf{U}} t \equiv \_)\ (\Gamma\ \Vdash_{\mathsf{U}} \_ : t)\ (\Gamma\ \Vdash_{\mathsf{U}} \_ \equiv \_ : t)\\
               & \sep & \Rnat : \forall\ \Gamma\ t \to
                          (\Gamma \Vdash_{\Nat} t) \to
                          \AgdaData{R}^\ell\ \Gamma\ t\ (\Gamma\ \Vdash_{\Nat} t \equiv \_)\ (\Gamma\ \Vdash_{\Nat} \_ : t)\ (\Gamma\ \Vdash_{\Nat
                          } \_ \equiv \_ : t)\\
               & \sep & \Rpi : \forall\ \Gamma\ t \to
                          (\Gamma \Vdash_{\Pi} t) \to
                          \AgdaData{R}^\ell\ \Gamma\ t\ (\Gamma\ \Vdash_{\Pi} t \equiv \_)\ (\Gamma\ \Vdash_{\Pi} \_ : t)\ (\Gamma\ \Vdash_{\Pi} \_ \equiv \_ : t)\\
               & \sep & \ROmega : \forall\ \Gamma\ t \to
                          (\Gamma \Vdash_{\Omega} t) \to
                          \AgdaData{R}^\ell\ \Gamma\ t\ (\Gamma\ \Vdash_{\Omega} t \equiv \_)\ (\Gamma\ \Vdash_{\Omega} \_ : t)\ (\Gamma\ \Vdash_{\Omega} \_ \equiv \_ : t)\\
               & \sep & \Rforall : \forall\ \Gamma\ t \to
                          (\Gamma \Vdash_{\forall} t) \to
                          \AgdaData{R}^\ell\ \Gamma\ t\ (\Gamma\ \Vdash_{\forall} t \equiv \_)\ (\Gamma\ \Vdash_{\forall} \_ : t)\ (\Gamma\ \Vdash_{\forall} \_ \equiv \_ : t)\\
               & \sep & \Rexists : \forall\ \Gamma\ t \to
                          (\Gamma \Vdash_{\exists} t) \to
                          \AgdaData{R}^\ell\ \Gamma\ t\ (\Gamma\ \Vdash_{\exists} t \equiv \_)\ (\Gamma\ \Vdash_{\exists} \_ : t)\ (\Gamma\ \Vdash_{\exists} \_ \equiv \_ : t)\\
               & \sep & \Rempty : \forall\ \Gamma\ t \to
                          (\Gamma \Vdash_{\bot} t) \to
                          \AgdaData{R}^\ell\ \Gamma\ t\ (\Gamma\ \Vdash_{\bot} t \equiv \_)\ (\Gamma\ \Vdash_{\bot} \_ : t)\ (\Gamma\ \Vdash_{\bot} \_ \equiv \_ : t)\\
               %% & \sep & \Remb : \forall\ \Gamma\ t\ P_=\ P_t\ P_{t=} \to \AgdaData{R}^{\ell'}\ \Gamma\ t\ P_=\ P_t\ P_{t=} \to \AgdaData{R}^\ell\ \Gamma\ t\ P_=\ P_t\ P_{t=} \qquad \text{for all } \ell' < \ell
  \end{array} &&
\end{flalign*}

\begin{flalign*}
& \text{All the functions whose name contains } \Vdash_{\mathsf{ne}} \, , \ \Vdash_{\mathsf{U}} \, , \ \Vdash_{\Nat} \, ,\ \Vdash_{\Pi} \, ,\ \Vdash_{\forall} \, ,\ \Vdash_{\exists} \, ,\ \Vdash_{\bot}
\text{ are as defined in \cref{sec:logic-relat-with}.} &
\end{flalign*}

\begin{flalign*}
  \begin{array}{lcl}
\_ \Vdash_\ell \_ : \_ & : & (\Gamma : \Context) \to (t : \Term) \to (A : \Term) \to (s : \Sort) \to \AgdaSet{\ell+1} \\
\Gamma \Vdash_\ell t : s & = & (P_= : \Term \to \AgdaSet{\ell}) \times (P_t : \Term \to \AgdaSet{\ell}) \\ && \times\ (P_{t=} : \Term \to \Term \to \AgdaSet{\ell}) \times (\AgdaData{R}^\ell\ \Gamma\ t\ s\ P_=\ P_t\ P_{t=})
  \end{array} &&
\end{flalign*}

\begin{flalign*}
  \begin{array}{lcl}
\_ \Vdash_\ell \_ \equiv \_ : \_ & : & (\Gamma : \Context) \to (A : \Term) \to (B : \Term) \to (s : \Sort) \to \{ \Gamma \Vdash_\ell A : s \} \to \AgdaSet{\ell} \\
\Gamma \Vdash_\ell A \equiv B : s & = & P_=\ B
  \end{array} &&
\end{flalign*}

\begin{flalign*}
  \begin{array}{lcl}
\_ \Vdash_\ell \_ : \_ : \_ & : & (\Gamma : \Context) \to (t : \Term) \to (A : \Term) \to (s : \Sort) \to \{ \Gamma \Vdash_\ell A : s \} \to \AgdaSet{\ell} \\
\Gamma \Vdash_\ell t : A : s & = & P_t\ t
  \end{array} &&
\end{flalign*}

\begin{flalign*}
  \begin{array}{lcl}
\_ \Vdash_\ell \_ \equiv \_ : \_ : \_ & : & (\Gamma : \Context) \to (t\ u : \Term) \to (A : \Term) \to (s : \Sort) \to \{ \Gamma \Vdash_\ell A : s \} \to \AgdaSet{\ell} \\
\Gamma \Vdash_\ell t \equiv u : A : s & = & P_{t=}\ t\ u
  \end{array} &&
\end{flalign*}
\end{small}
  \caption{Inductive encoding of the logical relation}
  \label{fig:logrel-ind}
\end{figure}

Constructing this universe of reducible types is only the first half
of the normalization proof. To complete the construction, we also need
to encode validity without induction-recursion in \( \MLTT \) and
prove the fundamental lemma.
%
We do not develop this here as it uses a similar construction and
refer the interested reader to the formalization for the full proof.

\section{Semantics of \SetoidCC}
\label{sec:cons-seto-model}

In this section, we turn to the construction of the
\emph{standard} set-theoretic model of \SetoidCC.
%
This model serves two purposes: to show consistency of our theory on
the one hand, and to ensure that \SetoidCC is a good internal language for
set theory on the other hand.

\subsection{Deriving Consistency from a Model}

As we already mentioned, the normalization model that we built in
\cref{sec:logic-relat-with} is not strong enough to obtain consistency
and canonicity for \SetoidCC. Indeed, this model interprets the
proof-irrelevant proposition \( \bot \) as the set of syntactical
terms with type \( \bot \), which does not grant any kind of control
over its proofs in the empty context.

This was already the case in the work of \sidecitet{pujet:hal-03367052} on
\SetoidCC, so they supplemented their normalization proof with a model
in the category of sets (presented as setoids), using
induction-recursion to interpret universes as sets of codes \emph{à
  la Tarski} defined mutually with eliminators and coercion functions.
%
In that model, \( \bot \) is actually interpreted as the empty set, which
proves consistency of \SetoidCC: we can use it to show that there is
no term of type \( \bot \) in the empty context.

\subsection{\SetoidCC as an Internal Language for Sets}

While the construction of \sidecitet{pujet:hal-03367052} does show consistency of
\SetoidCC, it arguably falls short of presenting \SetoidCC as an internal
language for classical mathematics.
%
Indeed, the inductive-recursive construction of the universe of codes is
designed to only account for the codes that come from the syntax of
\SetoidCC. In other words, the following statement
\[
\Pi(A : \Type)(x\ y\ z : A).\,\Fun{\Fun{\Obseq[A]{x}{y}}{\Obseq[A]{y}{z}}}{\Obseq[A]{x}{z}}
\]
only states transitivity of equality for sets that are built from
the type formers of \SetoidCC, when interpreted in their model.

This state of affairs is obviously not ideal, as readers who are not
willing to accept \SetoidCC as a new foundation of mathematics would
probably be more inclined to use it if they knew that any theorem they
proved in \SetoidCC is also true for classical set theory.
%
Thus, instead of replicating this construction in an impredicative
meta-theory (to be able to interpret \( \sProp \)), we seize the
opportunity to solve two problems at once, and present a model in classical
set theory that gives a much more sensible meaning to statements in \SetoidCC
while still being able to show consistency.

\subsection{Constructing the Set-theoretical Model}

We work in ZF set theory with a countable hierarchy of Grothendieck universes
\( \sV_0, \sV_1, ... \)
%
We call \( \Two \) the lattice of truth values (or in other words, the
booleans), with \( \independent \) as its minimal value,
and given \( p \in \Two \) we write \( \val\ p \) for the
corresponding sub-singleton set \( \{ x \in \{ * \}\ |\ p \} \).
%
Even though we stay in the world of set theory for the duration of
this section, we remain type theorists, and as such
we use dependent products, dependent sums and inductive types in this
section. They should be understood as their standard interpretation in
the set-theoretic model of type theory.
%
In an attempt to minimize confusion, we change our notations a little
bit: we use \( (a \in A) \to (B\ a) \) for the set-theoretic dependent product,
\( (a \in A) \times (B\ a) \) for the the set-theoretic dependent sum and
\( \metanat \) for the set-theoretic integers.

The central ingredient of our standard model is the hierarchy of universes of
codes \( \sU_0, \sU_1... \), that are used to interpret the proof-relevant universes
of \SetoidCC.
%
It is tempting to define them directly as Grothendieck universes
(\( \sU_i = \sV_i \)), but unfortunately this does not work: in \SetoidCC,
type constructors are injective with respect to the observational equality
(for instance, \( \Obseq{(\Fun{A}{B})}{(\Fun{A'}{B'})} \) implies that
\( \Obseq{A}{A'} \) and \( \Obseq{B}{B'} \)) while set-theoretic function
spaces collapse too much information for this to be possible
(\( \Fun{\bot}{\Bool} \) and \( \Fun{\bot}{\Nat} \) are identical).

This is not too difficult to fix, however: we simply need to add a bit of
information to the codes in the universe, so that we can keep track of how
we constructed any given set. The most natural way to do this from a type
theorist's perspective is to build an inductive predicate \( \ \Upred \) over
\( \sV_i \) and then interpret the universes of \SetoidCC as
\( \sU_i = (X \in \sV_i)\times(\Upred\ X) \), as described in \cref{fig:model}.
%
Note how the constructor \( \codeemb \) builds a proof of \( \ \Upred\ X \)
for any (small) set \( X \), so that statements that quantify over
\( \varType_i \) apply to all appropriately-sized sets of the model.
%
This also implies that there might be several codes for the
same set: for instance \( (\Fun{\Nat}{\Nat} \ ;\ \codeemb) \) and
\( (\Fun{\Nat}{\Nat} \ ;\ \codepiuu\ ...) \) are both codes for
the set \( \Fun{\Nat}{\Nat} \), but only the second one remembers that it
has been built as a function space.

Remark that the natural equality between codes matches closely the equality
of \SetoidCC: two codes can only be equal if they pack the same witness of
\( \Upred \), which means that they have been built in the same way.
Moreover, if we compare two codes of the form
\( (X \to Y ;\ \codepiuu\ ...) \), the equality of the second member means that
the domains and the codomains of the function spaces are equal---we recover the injectivity of
the type formers.

\begin{figure}
   \begin{small}
\begin{flalign*}
  \begin{array}{lcl}
   \Upred         & : & \Fun{\sV_i}{\sV_{i+1}}\\
   \Upred         & \bnfis & \tm{\codeemb}{\Fun{(X \in \sV_i)}{\Upred\ X}}\\
%                     & \sep & \tm{\codenat}{\sU_i} & \qquad & \text{if i = 0}\\
                     & \sep & \tm{\codepiuu}{\{ j, k \in \metanat\ |\ \mathrm{max}(j, k) \le i\} \to (A \in \sV_j) \to (A_\varepsilon \in \Upred\ A)} \\
                  & & \qquad \to (B \in (A \to \sV_k)) \to (B_\varepsilon \in ((a \in A) \to \Upred\ (B\ a)))\\
                  & & \qquad \to \Upred\ {((a \in A) \to B\ a)}\\
                     & \sep & \tm{\codepisu}{(A \in \ssProp)} \\
                  & & \qquad \to (B \in (\val\ A \to \sV_i)) \to (B_\varepsilon \in ((a \in \val\ A) \to \Upred\ (B\ a)))\\
                  & & \qquad \to \Upred\ {((a \in \val\ A) \to B\ a)}\\
                     & \sep & \tm{\codeU}{\Fun{\{ j \in \metanat\ |\ j < i\}}{\Upred\ \sU_j}} \\
                     & \sep & \tm{\codesProp}{\Upred\ \ssProp} \\
  \end{array} &&
\end{flalign*}

\begin{flalign*}
  \hspace{5pt}\sU_i \quad := \quad (X \in \sV_i)\times(\Upred\ X) &&
\end{flalign*}
%
\begin{flalign*}
  &\hspace{5pt}\el \quad : \quad \sU_i \to \sV_i&\\
  &\hspace{5pt}\el\ (X\ ;\ X_\varepsilon)\ :=\ X&
\end{flalign*}
\end{small}
  \caption{The universe of codes, defined with set-theoretic inductive types}
  \label{fig:model}
\end{figure}

\subsection{Interpreting the Syntax of \SetoidCC}

Now that we know how to deal with the universes, it only
remains to spell out the interpretation of the full syntax of \SetoidCC.
%
In the standard fashion~\sidecite{hofmann95}, we define interpretation functions that send
\begin{itemize}
  \item a context \( \Gamma \) to a set \( \mctx{\Gamma} \), and
  \item a pair of a term \( t \) and a context \( \Gamma \) to a set \( \mty{t} \) indexed over \( \mctx{\Gamma} \).
\end{itemize}
Since these functions are defined on raw syntax without any guarantee of
well-typedness, we cannot expect every term to have a sensible interpretation
so we need to make them \emph{partial} functions. We will be able to prove
that all well-typed terms admit an interpretation \textit{a posteriori} by
induction on the derivations.

As usual, contexts are interpreted as telescopes:
\( \mctx{\emptyctx} = \{ * \} \) and
\( \mctx{\extctx{\Gamma}{A}} = (\rho : \mctx{\Gamma})\times (\mty{A}\ \rho) \)
when \( \mty{A} \) is defined.
%
Given \( \rho \in \mctx{\Gamma} \) and a variable \( x \) that appears in
\( \Gamma \), we write \( \rho(x) \) for the corresponding projection.
%
The interpretation of the terms is defined in
\cref{fig:interpretation}. The proof-relevant fragment unsurprisingly
follows the standard interpretation of dependent type theory, and the
observational equality is interpreted as the extensional equality of
set-theory.
%
Note how the irrelevant proofs are all interpreted as \( * \), the inhabitant
of the singleton set. This forces us to interpret cast as doing nothing, and \( \emptyrec{A}{t} \) is not even interpreted since it can only be formed in inconsistent contexts, which are empty in the model.

In order to prove the soundness of our interpretation, we need to extend it to weakenings
and substitutions between contexts.
%
Assume \( \Gamma \) and \( \Delta \) are a syntactical contexts, and \( A \) and \( t \)
are syntactical terms.
%
In case \( \mctx{\Gamma, \tm{x}{A}, \Delta} \) and \( \mctx{\Gamma, \Delta} \) are well-defined,
let \( \pi_A \) be the projection:
{\small
\[
  \begin{split}
  \pi_A : \mctx{\Gamma, \tm{x}{A}, \Delta} & \to \mctx{\Gamma, \Delta} \\
  (\vec{x_\Gamma}, x_A, \vec{x_\Delta}) & \mapsto (\vec{x_\Gamma}, \vec{x_\Delta}).
  \end{split}
\]
}
In case \( \mctx{\Gamma, \subst{\Delta}{t}} \) and \( \mctx{\Gamma, \tm{x}{A}, \Delta} \) are
well-defined, we define the function \( \sigma_t \) by:
{\small
\[
  \begin{split}
    \sigma_t : \mctx{\Gamma, \subst{\Delta}{t}} & \to \mctx{\Gamma, \tm{x}{A}, \Delta} \\
    (\vec{x_\Gamma}, \vec{x_\Delta}) & \mapsto (\vec{x_\Gamma}, \mty{t}\ \vec{x_\Gamma}, \vec{x_\Delta}).
  \end{split}
\]
}
%
\begin{lemma}[Weakening]\label{lem:weakening}
  \( \pi_A \) is the semantic counterpart to the weakening of \( A \): for all terms \( u \),
  when both sides are well defined, we have:
  {\small
\[
    \mty[{\Gamma, \tm{x}{A}, \Delta}]{u} = \mty[{\Gamma, \Delta}]{u} \circ \pi_A
  \]}
\end{lemma}
%
\begin{lemma}[Substitution]\label{lem:substitution}\ \( \sigma_t \) is the semantic counterpart to the substitution by \( t \):
  for all terms \( u \), when both sides are well defined, we have:
  {\small
\[
    \mty[{\Gamma, \subst{\Delta}{t}}]{\subst{u}{t}} = \mty[{\Gamma, \tm{x}{A}, \Delta}]{u} \circ \sigma_t
  \]}
\end{lemma}

\begin{theorem}[Soundness of the Standard Model]
  \
  \begin{enumerate}
    \item If\, \( \wfctx{\Gamma} \) then \( \mctx{\Gamma} \) is defined.
    \item If\, \( \tytm{\Gamma}{A}{\sProp} \) then \( \mty{A}\ \rho \) is a proposition for all \( \rho \in \mctx{\Gamma} \).
    \item If\, \( \tytm{\Gamma}{A}{\varType_i} \) then \( \mty{A}\ \rho \) is in \( \sU_i \) for all \( \rho \in \mctx{\Gamma} \).
    \item If\, \( \tytmannotated{\Gamma}{t}{A}{\sProp} \) then \( \mty{A}\ \rho \) is true for all \( \rho \in \mctx{\Gamma} \).
    \item If\, \( \tytmannotated{\Gamma}{t}{A}{\varType_i} \) then \( \mty{t}\ \rho \) is in \( \el\ (\mty{A}\ \rho) \) for all \( \rho \in \mctx{\Gamma} \).
    \item If\, \( \eqtm{\Gamma}{t}{u}{A} \) then \( \mty{t} = \mty{u} \).
  \end{enumerate}
\end{theorem}
\begin{proof}
  By induction on the typing derivations, using \cref{lem:weakening,lem:substitution}.
\end{proof}


\begin{figure}
  \begin{small}
\[
\begin{array}{rcl}
  \mty{x}\ \rho & := & \rho(x) \\
  \mty{\varType_i}\ \rho & := & \langle \sU_i\ ;\ \codeU\ i \rangle \\
  \mty{\sProp[i]}\ \rho & := & \langle \ssProp\ ;\ \codesProp \rangle\\
  \mty{\Depfunannotated[y]{F}{G}{}{\varType_j}{\varType_k}}\ \rho & := & \big\langle (x \in\el\ \mty{F}\ \rho)\to (\el\ \mty[{\extctx[x]{\Gamma}{F}}]{G}\ \langle \rho\ ;\ x\rangle)\\
& & ;\ \codepiuu\ j\ k\ (\mty{F}\ \rho)\ (x \mapsto \mty[{\extctx[x]{\Gamma}{F}}]{G}\ \langle \rho\ ;\ x\rangle) \big\rangle\\
  \mty{\Depfunannotated[y]{F}{G}{}{\sProp}{\varType_k}}\ \rho & := & \big\langle (x \in\val\ \mty{F}\ \rho)\to (\el\ \mty[{\extctx[x]{\Gamma}{F}}]{G}\ \langle \rho\ ;\ x\rangle)\\
& & ;\ \codepisu\ (\mty{F}\ \rho)\ (x \mapsto \mty[{\extctx[x]{\Gamma}{F}}]{G}\ \langle \rho\ ;\ x\rangle) \big\rangle\\
  \mty{\Depfunannotated[y]{F}{G}{}{\varType_j}{\sProp}}\ \rho & := & \forall x \in (\el\ \mty{F}\ \rho),\ \mty[{\extctx[x]{\Gamma}{F}}]{G}\ \langle \rho\ ;\ x\rangle\\
  \mty{\Depfunannotated[y]{F}{G}{}{\sProp}{\sProp}}\ \rho & := & (\mty{F}\ \rho) \Rightarrow (\mty[{\extctx[x]{\Gamma}{F}}]{G}\ \langle \rho\ ;\ *\rangle)\\
  \mty{\lam[x]{F}{t}}\ \rho & := & x \mapsto (\mty[{\extctx[x]{\Gamma}{F}}]{t}\ \langle \rho\ ;\ x\rangle)\\
  \mty{t\ u}\ \rho & := & (\mty{t}\ \rho)\ (\mty{u}\ \rho)\\
  \mty{\Nat}\ \rho & := & \codeemb\ \metanat \\
  \mty{\zero}\ \rho & := & \metazero \\
  \mty{\suc{t}}\ \rho & := & \metasuc{(\mty{t}\ \rho)} \\
  \mty{\natrec{P}{t_0}{t_S}{n}}\ \rho & := & \metanat\mathrm{-elim}(\el \circ (\mty{P}\ \rho), \mty{t_0}\ \rho, \mty{t_S}\ \rho, \mty{n}\ \rho)\\
  \mty{\Exists[y]{F}{G}}\ \rho & := & (\mty{F}\ \rho) \land (\mty[{\extctx[y]{\Gamma}{F}}]{G}\ \langle \rho\ ;\ * \rangle) \\
  \mty{\pair{t}{u}}\ \rho & := & * \\
  \mty{\fst{t}}\ \rho & := & * \\
  \mty{\snd{t}}\ \rho & := & * \\
  \mty{\Empty}\ \rho & := & \independent \in \Two \\
  \mty{\emptyrec{A}{t}}\ \rho & := & \text{undefined} \\
  \mty{\Obseq[A]{t}{u}}\ \rho & := & \sEq{\mty{t}\ \rho}{\mty{u}\ \rho} \qquad\text{in }\el(\mty{A}\ x) \\
  \mty{\refl{t}{A}}\ \rho & := & * \\
  \mty{\transport{F}{t}{G}{u}{t'}{e}}\ \rho & := & * \\
  \mty{\cast{A}{B}{e}{t}}\ \rho & := & t\\
  \mty{\castrefl{A}{t}}\ \rho & := & *
\end{array}
\]
\end{small}
  \caption{Interpretation of \SetoidCC in the Standard Model}
  \label{fig:interpretation}
\end{figure}

\subsection{Consequences of the Model}

From the soundness theorem and the interpretation of \( \bot \) being empty,
the consistency of \SetoidCC is immediate:

\begin{theorem}[Consistency]
  There are no proofs of \( \bot \) in the empty context.
\end{theorem}

Furthermore, by inspecting the normal forms provided by the normalization
theorem, we realize that \( \castName \) is the only way to build a neutral
term in the absence of variables. But having a stuck \( \castName \) requires
having an equality proof between two incompatible types, which contradicts
consistency.

\begin{theorem}
  There are no neutral terms in the empty context.
\end{theorem}

From there, we deduce the canonicity theorem: all integers reduce to
standard integers in the empty context. Thus, our modified model still allows
us to establish all of the important meta-theoretical properties.

On the other hand, it seems difficult to measure to what extent our
set-theoretic model presents
\SetoidCC as a good internal language for sets. Compared to the universe
construction of \sidecitet{pujet:hal-03367052}, we gain the property that every set belongs
to a universe---meaning that theorems which quantify over \( \varType_i \)
will apply to all sufficiently small sets.
%
Of course, this does not prevent us from proving some theorems that are
obviously false in classical mathematics, such as the
injectivity of type formers. But we would argue that all such results
are artifacts of the choice of encodings, and not meaningful
mathematical statements.

