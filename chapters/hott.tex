\setchapterimage[6cm]{seaside}
\setchapterpreamble[u]{\margintoc}
\chapter{The cubical model of Univalence}
\labch{hott}

\section{Homotopy Type Theory}

Homotopy type theory (HoTT) is a new foundation that has emerged from
recently discovered connections between type theory and homotopy
theory~\sidecitep{HottBook13}. These foundations are based on the
observation that the inductively defined \emph{equality types} in type
theory behave like \emph{paths} in homotopy theory. On the one hand,
HoTT provides type theory with extensionality principles that were
previously not typically available, including function and
propositional extensionality (pointwise equal functions and logically
equivalent propositions are equal). These principles follow from a
more general extensionality principle called \emph{univalence}. This
principle was originally proposed by Vladimir Voevodsky in the form of
his celebrated \emph{univalence axiom}~\sidecite{Voevodsky10} that can be
consistently assumed in type theory~\sidecitep{KapulkinLumsdaine12}. On
the other hand, HoTT provides methods for reasoning formally about
homotopical notions in a synthetic manner. This is made possible
through \emph{higher inductive types} (HITs) that allow spaces to be
directly represented as types with nontrivial equality/path
constructors.

\section{lol}

The other line of work is more recent and takes its roots in the
formulation of the univalence axiom~\sidecite{kapulkin2012simplicial,hottbook},
which gives a new meaning to the equality between types: two types are
equal when they equivalent. This can be understood as an extensionality
principle for the universe of types.
%
In their search for a computational interpretation of
univalence, \sidecitet{cubicaltt} developed a notion of cubical equality,
which satisfies funext and univalence, but is incompatible with UIP.

Thus, observational equality and cubical equality are two diverging
directions for providing propositional equality with more extensionality.
\sidecitet{hts-sota,Altenkirch2016ExtendingHT,Capriotti17} advocate that the
two solutions are actually complementary and can be integrated to
a single system with two universe hierarchies, one that satisfies
univalence and the other that satisfies UIP.


