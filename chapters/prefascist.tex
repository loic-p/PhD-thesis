\setchapterimage[6cm]{seaside}
\setchapterpreamble[u]{\margintoc}
\chapter{Prefascist Types}
\labch{prefascist}

As we conclude our explorations in the world of observational type theory 
and proof-irrelevant equalities, we turn to the radically different world 
of the univalent equality and of the cubical models of univalent type 
theory~\cite{BezemCoquandHuber14, CCHM}. We start our new journey with
a chapter about \emph{presheaves} in intensional type theory.

Presheaves are a versatile object in category theory that plays a significant
role in the meta-theory of dependent type theory. 
In particular, presheaves form the basis of the cubical models of univalent type
theory which will be the focus of the next chapter.
Thus in this preparatory chapter, we discuss the relation between presheaves and 
intensional type theory.

We start this chapter with a short overview of the basic properties of presheaves in
set theory in \cref{sec:categorical-reminders}. Then in 
\cref{sec:intensional-problems}, we explain some well-known difficulties
with defining presheaves of types in intensional type theory. In
\cref{sec:prefascist} we introduce prefascist types, an alternative
definition of presheaves in type theory proposed by 
Pédrot~\sidecite{pedrot:prefascist} to build models of type theory in
type theory.

Given the nature of the topic, the reader should expect significantly more
category theory in this chapter than in the rest of this thesis. 
We assume some amount of familiariaty with objects such as functors, limits, 
or adjunctions.

\section{Set-theoretical Presheaves}
\label{sec:categorical-reminders}

In this section, we recall the definition of presheaves and go 
over some of their uses.
% 
The reader who is already well acquainted with presheaves might want to skip
this section and go directly to \cref{sec:intensional-problems}, where we
investigate presheaves from the angle of intensional type theory.
% 
But until then, we will be working informally in a constructive set theory with
\defnote{Grothendieck universes}{Grothendieck universes allow us to safely 
manipulate objects such as the set of all \emph{small} sets, which is a 
\emph{large} set. Grothendieck universes work very much like the predicative 
universe hierarchy in type theory.}.

Given any category \( \cat{C} \), the category 
\( \catPsh{\cat{C}} \) of \emph{presheaves} on \( \cat{C} \) is the category 
of functors from the opposite category \( \op{\cat{C}} \) to the category of 
small sets.
% 
\begin{definition}
\(
    \catPsh{\cat{C}} := \hom[\catCat]{\op{\cat{C}}}{\catSet}
\)
\end{definition}
% 
In this definition, the \( \mathrm{Hom}_\catCat \) notation should 
be understood in the 2-categorical sense: this object is a category, not a set.
% 
In other words, a presheaf is a functor from \( \op{\cat{C}} \) to \( \catSet \), 
and a morphism of presheaves is a natural transformation.

In the usual fashion of category theory, this definition is slick
and abstract, which results in a remarkably versatile object. 
% 
But in counterpart, the definition alone is not too helpful in building
intuition about presheaves---what are they good for?

Presheaves\sidenote{sheaves, actually} were initially designed to capture the
idea of a set continuously parametrized by a topological space \( X \), by taking the 
lattice of open subsets of \( X \) as the index category \( \cat{C} \). 
% 
The definition was later generalized to an arbitrary category, but this 
topological legacy has left traces in the terminology, with words such as 
{\color{defcolor} section} and \defnote{restriction}{If \( F \) is a presheaf 
on \( \cat{C} \), then the elements of \( F(p) \) are called sections of 
\( F \) over \( p \), and the image of a morphism \( F(\alpha) \) is called 
the restriction along \( \alpha \).}.
% 
However in this thesis we will mostly be interested in presheaves different 
reasons, one of them being that presheaves allow us to freely add colimits to 
a category.

\subsection{Presheaves as a Categorical Cocompletion}

Suppose that \( \cat{C} \) is \defnote{locally small}{A category is
locally small if all of its hom-sets are small sets}.
Recall the definition of the Yoneda embedding \( \yo \):
\begin{definition}
The Yoneda embedding is a functor from \( \cat{C} \) to \( \catPsh{\cat{C}} \)
defined by
\[
\begin{array}{lcl}
    \yo & : & \cat{C} \to \catPsh{\cat{C}} \\
    \yo(c) & := & \hom[\cat{C}]{\_}{c}
\end{array}
\]
The action of \( \yo \) on morphisms is given by post-composition, which
makes it into a proper functor from \( \cat{C} \) to its presheaf
category \( \catPsh{\cat{C}} \). 
\end{definition}
% 
Furthermore, the Yoneda lemma teaches us that this functor is fully faithful, 
and thus the \defnote{representable}{A presheaf is called representable when 
it is in the image of \yo} presheaves can be understood as a copy of 
\( \cat{C} \) sitting inside the larger category \( \catPsh{\cat{C}} \).

There is a precise sense in which \( \catPsh{\cat{C}} \) is generated from 
this internal copy of \( \cat{C} \) by freely adding all small colimits:

\begin{theorem}
    The category \( \catPsh{\cat{C}} \) admits all small colimits, which can
    be computed point-wise.
\end{theorem}

\begin{theorem}
    Any object of \( \catPsh{\cat{C}} \) can be obtained as a small colimit of 
    representable presheaves.
\end{theorem}

\begin{theorem}
    Given a category \( \cat{D} \) that has all small colimits and a functor
    \( F : \cat{C} \to \cat{D} \), there is a unique cocontinuous functor
    \( \widehat{\func{F₀}} : \catPsh{\cat{C}} \to \cat{D} \) such that 
    \( F = \widehat{\func{F₀}} \circ \yo \).
\end{theorem}

Proofs can be found in \sidecitet{MaclaneMoerdijk}.

Put together, these three theorems seem to paint a surprisingly simple picture 
of \( \catPsh{\cat{C}} \): we can think of a presheaf as an amalgamated sum of 
objects from \( \cat{C} \).
% 
This is why combinatorial models of spaces such as \emph{simplicial sets} and 
\emph{cubical sets} are defined as presheaves on carefully chosen categories. 
We will come back to this in the next chapter.

However, mental images are a double edged sword and can be just as misleading
as they are enlightening. In particular, the reader should note that the yoneda 
embedding does not preserve colimits that may already exist in \( \cat{C} \),
and as a consequence we should be careful to have different pictures for
\( \yo(x+y) \) and \( \yo(x) + \yo(y) \).

\subsection{Presheaves as Generalized Categories of Sets}

In addition to being cocomplete, categories of presheaves also have small
limits and exponential objects~\sidecite{MaclaneMoerdijk}.

\begin{theorem}
    The category \( \catPsh{\cat{C}} \) admits all small limits, which can
    be computed point-wise. Moreover, the Yoneda embedding preserves
    limits that exist in \( \cat{C} \).
\end{theorem}

\begin{theorem}
    The category \( \catPsh{\cat{C}} \) is cartesian closed.
\end{theorem}

This is only natural: presheaves are more or less indexed small sets, and as 
such they retain a large part of the structure of the category \( \catSet \). In fact, they
retain much more structure than just limits and colimits: we can basically 
% 
\sideremark{The category of presheaves on \( \cat{C} \) is the prototypical example of a \emph{topos}, 
 a category that shares most of the abstract categorical properties of \( \catSet \).}
% 
replicate any construction from constructive set theory inside a category of 
presheaves---we can form powersets, the image of a morphism, dependent 
products, etc. However, the principle of excluded middle and the axiom of 
choice are not valid in most categories of presheaves.

In particular, presheaves have enough structure to interpret 
dependent products, dependent sums and inductive types. As a consequence, we
can build models of Martin-Löf type theory in categories of 
presheaves~\sidecite{Hofmann97}.

\section{Type-theoretical Presheaves}
\label{sec:intensional-problems}

Hopefully by now, the reader is convinced that presheaves are a simple yet 
versatile idea in category theory and thus a natural candidate for a 
translation into type theory.
% 
But while concrete first-order objects such as integers or finite trees 
are usually pretty straightforward to implement in type theory, higher-order objects 
that deal with predicates or functions usually come with their lot of 
complications---and presheaves are no exception.

In this section, we explain the difficulties with defining presheaves of types
in intensional Martin-Löf Type Theory. 
% 
We use the syntax of the \Agda proof assistant for our type theoretical 
definitions, without assuming Streicher's axiom K.

\subsection{First Definition}

Let us assume that we are working with a small category \( \cat{C} \), and that we
managed to encode it nicely inside type theory:
\sideremark[*2]{Arguments between curly braces are optional arguments, which will
    usually be inferred from the context.}

\ExecuteMetaData[chapters/literate-agda/Presheaf.tex]{category}

Furthermore, we assume that the unit laws and the associativity of
composition in \( \cat{C} \) hold \emph{by definition}:

\ExecuteMetaData[chapters/literate-agda/Presheaf.tex]{equations}

\sideremark{In this chapter, we use the \Agda conventions. This means that
    we use a triple equal symbol \func{≡} for the propositional equality.
    It has nothing to do with the definitional equality of \cref{ch:observational}
    or the reducible equality of \cref{ch:metatheory}. The definitional equality
    of \Agda is written with the usual equality symbol \( {=} \).}
% 
By which we mean that the equality that appears in these three postulates is 
not the propositional equality (\( {\func{≡}} \)) but rather the definitional 
equality (\( {=} \)): the two sides are convertible.

Of course, using the definitional equality in \Agda code like this is not
allowed.
% 
This should rather be understood as a notational shorthand for the assumption
that we have an actual implementation of \( \cat{C} \) with instances of 
\func{Hom}, \func{id} and \func{\_∘\_} that satisfy these three definitional 
equations.
% 
It could be the case for instance if we managed to express \( \cat{C} \) as a 
full subcategory of \func{Set}, in which case the morphisms are just 
functions, whose composition is associative by defintion.
% 
This might seem like an arbitrary and restrictive constraint on the category 
\( \cat{C} \), but in \cref{sec:cat-strictification} we will see that all small 
categories can in fact be presented in this way in a type theory with strict 
propositions.

Now, how should we go about defining the type of presheaves on \( \cat{C} \)?
If we mimic the set-theoretic definition of a presheaf, we might write 
something that starts like this:

\ExecuteMetaData[chapters/literate-agda/Presheaf.tex]{presheaf1}

This defines the action of the presheaf on objects and morphisms, but we also need to verify some 
equations, as functors are supposed to preserve identities and composition.
% 
We can try to state them using the propositional equality \func{\_≡\_}:

\ExecuteMetaData[chapters/literate-agda/Presheaf.tex]{presheaf2}

and from there, all reasonable equalities such as
\begin{equation}\label{eq-psh}
    \field{F₁}\ ((f\ \func{\circ}\ g)\ \func{\circ}\ h) 
    \ \func{≡}\ 
    (\field{F₁}\ h\ \func{\circ}\ \field{F₁}\ g)\ \func{\circ}\ \field{F₁}\ f
\end{equation}
can be proved through successive rewritings using \field{F\_id} and 
\field{F\_comp}.
% 
And sure enough, this definition does correspond to presheaves when we interpret
type theory in the usual set-theoretical model \sidecite{Hofmann97}. 
% 
But as bare \MLTT lacks the extensional features of set theory, this
definition hides some kinks that we might want to iron out.

\subsection{A Story about Higher Coherences}

Interestingly, if we try to prove equation \ref{eq-psh}, we can devise two
chains of rewritings that take the left hand side to the right hand
side:

\begin{center}
    \sideremark{The horizontal equalities are instances of \field{F\_comp},
        and the two vertical equalities hold by definition.}
\tikzset{triple/.style={-,AgdaDatatype,preaction={draw,AgdaDatatype,double,double distance=2pt}}}
\begin{tikzcd}
    \field{F₁}\ ((f\ \func{\circ}\ g)\ \func{\circ}\ h)
    \ar[triple]{r} \ar[equal]{d}
    & \field{F₁}\ h\ \func{\circ}\ \field{F₁}\ (f\ \func{\circ}\ g) 
    \ar[triple]{r}
    & \field{F₁}\ h\ \func{\circ}\ (\field{F₁}\ g\ \func{\circ}\ \field{F₁}\ f)
    \ar[equal]{d} \\
    \field{F₁}\ (f\ \func{\circ}\ (g\ \func{\circ}\ h))
    \ar[triple]{r}
    & \field{F₁}\ (g\ \func{\circ}\ h) \ \func{\circ}\ \field{F₁}\ f
    \ar[triple]{r}
    & (\field{F₁}\ h\ \func{\circ}\ \field{F₁}\ g)\ \func{\circ}\ \field{F₁}\ f
\end{tikzcd}
\end{center}

and these two chains of equalities result in two different witnesses of equation \ref{eq-psh}.
% 
We might hope to prove that these two witnesses are in fact equal ; but it is 
not possible without assuming additional axioms (remember that we are 
working in intensional \MLTT here, where identity proofs are by no means 
unique!).
% 
This state of affairs is rather unsatisfying, and it certainly looks like that 
kind of nuisance that will show up uninvited in our proofs later on.

To understand the source of this irritating phenomenon, it is helpful to look at 
it under the lens of Homotopy Type Theory.
% 
As we explained in \cref{ch:intro_english}, in \HoTT we think of types
as being \emph{spaces} instead of sets, and the two different notions of equality 
(definitional and propositional) have very different interpretations:
\begin{itemize}
    \item the definitional equality \( a = b \) corresponds to an actual, 
        on-the-nose equality between the two points, but
    \item the propositional equality \( a\ \func{≡}\ b \) is the space of
        paths connecting \( a \) to \( b \). 
\end{itemize}
% 
Since \MLTT is a subset of \HoTT, our tentative definition can be 
interpreted as talking about spaces and should ideally make sense
from this perspective. 
% 
Our definition would in fact work for presheaves of spaces if \field{F\_id} 
and \field{F\_comp} were definitional equalities,
% 
but when we relax these equalities to mere paths (propositional equalities) we 
are effectively adding superfluous information to the type---as we have seen, 
there are now several ways to use \field{F\_id} and \field{F\_comp} to prove 
what should really be a equality ``on the nose'' in the corresponding spaces.
% 
Thus instead of presheaves of spaces, we are defining something like presheaves
of spaces with superfluous free loops attached to them.

Of course, we do not have access to the definitional equality internally 
in type theory, so we have to find another way to eliminate these extra paths.
% 
We can try to add new fields to the record \func{presheaf} that enforce 
propositional equations between the redundant equality witnesses, as follows:

\begin{code}
\>[0] \field{F\_coh} : \AgdaSymbol{∀}\ \AgdaBound{f}\ \AgdaBound{g}\ \AgdaBound{h}\ \AgdaSymbol{→}
\>[94I]\ (\field{F\_comp}\ \AgdaBound{h}\ (\AgdaBound{f}\ \func{∘}\ \AgdaBound{g}))
\ \func{∙}\ ((\field{F₁}\ \AgdaBound{h})\ \func{∘}\ (\field{F\_comp}\ \AgdaBound{g}\ \AgdaBound{f}))\<%
\\
\>[.][@{}l@{}]\<[94I]%
\>[12]\func{≡}\ (\field{F\_comp}\ (\AgdaBound{g}\ \func{∘}\ \AgdaBound{h})\ \AgdaBound{f})
\ \func{∙}\ ((\field{F\_comp}\ \AgdaBound{h}\ \AgdaBound{g})\ \func{∘}\ (\field{F₁}\ \AgdaBound{f})) \<%
\end{code}
\sideremark[*-5]{\func{\_∙\_} denotes the transitivity of equality / composition of 
paths.}

Indeed, it is possible to find a finite number of equations that account for 
all superfluous paths introduced by \field{F\_id} and \field{F\_comp}.  
But this approach will not get us very far: 
% 
\field{F\_coh} introduces superfluous paths between paths, which results in
a type of presheaves with superfluous 2-dimensional loops.
% 
Going further in this direction only leads us towards an infinite tower 
of equalities of ever increasing dimensions, with no apparant way to make it 
fit into a finite definition.

Here, we recognize a well-known pattern, as difficulties with infinite 
towers of coherences show up for numerous definitions in Homotopy Type Theory.
% 
The most famous such definition is that of \emph{semi-simplicial types}, 
which was identified in the early years of \HoTT \cite{IAS2012semisimplicial}, 
and remains an open problem as of today \sidecite{buchholtzSemiSimplicial}.

Semi-simplicial types are presheaves on the category of finite ordinals and 
strictly increasing maps \( \Delta_+ \).
% 
As we will explain shortly, if our type theory supports strict propositions,
\( \Delta_+ \) can be presented as a category that satisfies the composition
and the unit laws by definition, which means that the definition of 
semi-simplicial sets is a special case of the problem we are trying to solve.
% 
This is bad news!

\subsection{Possible Workarounds}

Thanks to the perspective of Homotopy Type Theory, we now understand that
defining presheaves of types on a general category is delicate because of the 
intricate higher-dimensional structure that may be encoded in identity types.
% 
Even though the problem of simplicial types remains open, there are well-known 
ways to avoid coherence problems with presheaves in type theory.

\paragraph{Presheaves of hsets} If types in \HoTT correspond to spaces, there 
is a certain subclass of types that can be identified as sets.
% 
These are the \emph{hsets}: types \( X \) equipped with a proof that all 
paths between any two points of \( X \) are equal

\ExecuteMetaData[chapters/literate-agda/Presheaf.tex]{hset}

Interestingly, the \field{is\_hset} constraint removes all the non-trivial 
loops from \field{X} without introducing any higher-dimensional coherence 
problems (\sidecite{hottbook}, theorem 7.1.7). 
% 
Such is the mystery of coherence towers: sometimes we can find a finite 
definition that makes all higher coherence issues vanish in one go, and 
sometimes this definition remains elusive.

Anyway it is straightforward to define presheaves of hsets, and they can be used 
to develop large swaths of category theory without fearing coherence 
issues~\sidecite{hottcoq}. 
However these proofs are less general than what we could hope for if we had a 
working definition for presheaves of spaces, because they only apply to
the subclass of presheaves where all types are hsets.

\paragraph{Uniqueness of Identity Proofs (UIP)}
% 
Going further in that direction, we can modify the type theory itself to 
enforce the uniqueness of identity proofs, so that every type is a hset.
% 
For instance, in the observational type theory \SetoidCC from 
\cref{ch:observational} identity types are strict propositions,
which means all identity proofs are equal by definition.

\SetoidCC gets very close to a type theoretical account of constructive set 
theory, so we can reasonably expect that doing mathematics with presheaves 
will not pose significant problems.

\paragraph{Two-level type theories}
% 
Instead of curtailing the identity types, two-level type theories 
feature two distinct propositional equalities: a homotopical equality that 
encodes paths, as well as a strict equality that encodes the actual equality
of points and satisfies UIP~\sidecite{voevodsky:2013:hts, annenkov-capriotti-kraus:2017}.

In such a framework, we can write a definition for presheaves using the strict equality
and avoid coherence problems, while still supporting types with arbitrarily
complex homotopical structures---at the price of additional complexity
and manipulations of \emph{fibrancy conditions}.

\section{Prefascist Types}
\label{sec:prefascist}

All the definitions of presheaves that we just listed follow the same general
pattern: they keep the shape of the set-theory-inspired definition, and
then they find a way to force the propositional equalities involved in the
definition to behave like strict equalities, \ie to prevent them from 
introducing relevant path information. 

However, this is not the only possible approach: 
% 
in a 2020 paper~\sidecite{pedrot:prefascist}, Pédrot designs a syntactical model of \MLTT in
\( \MLTT+\varsProp \) such that every type is interpreted as a presheaf which
is functorial up to \emph{definitional equality}. 
% 
In order to do so, Pédrot introduces \defnote{prefascist types}{Prefascist 
types owe their eyebrow-raising name to the French word for presheaf, 
\textit{préfaisceau}, and to the humor of Pierre-Marie Pédrot.}, an original
take on the definition of a presheaf.

In this section, we explain the definition of prefascist types and show some of 
their properties.
% 
As with \cref{sec:intensional-problems}, we assume that we are working with a 
\emph{strict} category, that is a category \( \cat{C} \) that satisfies the 
associativity and unit laws by definition. 
% 
\sideremark{In \cref{ch:intro_english} we used \( \varsProp \) for the sort
 of strict propositions, and we reserved \( \varProp \) for the sort of 
 non-strict propositions of \Coq. But since we adopt \Agda's conventions
 in this chapter, we will use \func{Prop} for strict propositions.}
% 
Moreover, we will extend \MLTT with a sort of strict propositions \func{Prop},
as implemented by \Coq, \Agda and \Lean \cite{gilbert:hal-01859964}.
% 
It works just like the universe \func{Set}, except that any type \( A \) 
in \func{Prop} is proof-irrelevant, meaning that all terms with type 
\( A \) are convertible.
% 
We also add a strict truncation operator \func{∥\_∥ₛ} that embeds 
proof-relevant types in \func{Prop}.

As an illustration for the use of \func{Prop}, we can use it to define the category
\( \Delta_+ \) of finite ordinals and increasing maps as a strict category.
If we define the monotonicity predicate as a strict proposition, then 
morphisms of \( \Delta_+ \) are a pair of a function and a computationally
irrelevant monotonicity proof, so that composition reduces to regular function 
composition, which is definitionally associative.
% 
In \cref{sec:cat-strictification}, we will explain that we can handle all small
categories in this way.

\subsection{What are Prefascist Types?}

In order to arrive at a definition of prefascist types, we start by 
investigating strict presheaves of types, \ie presheaves of types that satisfy 
the functoriality equations by definition. 
% 
Even though we do not have an internal definition for their type, we can still 
assume we have an object \( F \) that satisfies the functoriality equations up 
to definitional equality:

\sideremark{Once again, it is impossible to use the definitional equality in
\Agda syntax. 
This should really be read as assuming actual implementations of \func{F₀} and 
\func{F₁} that satisfy the two equations \func{F\_id} and \func{F\_comp} by 
definition.}
\ExecuteMetaData[chapters/literate-agda/PrePrefascist.tex]{strictpsh}

Although we cannot define it internally in \MLTT, we will write 
\( \catsPsh{\cat{C}} \) for the category of strict type-theoretic presheaves
and strict natural transformations.
An element of \( \catsPsh{\cat{C}} \) is given by two terms 
\func{F₀} and \func{F₁} of \MLTT that satisfy the equations \func{F\_id} and 
\func{F\_comp} by definition, and a morphism is defined in a similar way.
Note that our category \( \catsPsh{\cat{C}} \) of type-theoretic presheaves 
only contains \emph{internal presheaves}: we do not consider arbitrary 
set-theoretic functors from \( \cat{C} \) to the
category of types, but only functors that can be expressed in \MLTT.

Now let us assume that we are working with specific instances of \( \func{F₀} \)
and \( \func{F₁} \).
We remark that given \( x \) in \( \func{F₀}\ a \), we can use \( \func{F₁} \) to 
get an inhabitant of \( \func{F₀}\ b \) whenever we have a morphism 
from \( b \) to \( a \).
% 
In other words, having the presheaf structure laying around means that elements
of \( \func{F₀}\ a \) are really promoted to elements of the ``completed'' type family
\( \widehat{\func{F₀}} \), which is defined as follows:
\[
\begin{array}{lcl}
\widehat{\func{F₀}} & : & (a : \func{Obj}) \to \func{Set} \\
\widehat{\func{F₀}}\ a & := & \AgdaSymbol{∀}\ b\ (f : \func{Hom}\ b\ a)\ \AgdaSymbol{→}\ \func{F₀}\ b
\end{array}
\]
And the promoted version of \( x : \func{F₀}\ a \) is given by
\[
\widehat{x} := \AgdaSymbol{λ}\ b\ f\ \AgdaSymbol{→}\ \func{F₁}\ f\ x.
\]
Tracking all restrictions like this allows us to exhibit a natural presheaf 
structure for \( \widehat{\func{F₀}} \) by defining functoriality with a 
simple composition. If \( f \) is a morphism from \( b \) to \( a \), we define
\[
\begin{array}{lcl}
    f\ \func{∙}\ \_ & : & \widehat{\func{F₀}}\ a\ \AgdaSymbol{→}\ \widehat{\func{F₀}}\ b \\
    f\ \func{∙}\ \widehat{x} & := & \AgdaSymbol{λ}\ c\ g\ \AgdaSymbol{→}\ \widehat{x}\ c\ (f\ \func{∘}\ g).    
\end{array}
\]
Now, remember how we insisted that \( \cat{C} \) satisfies associativity and 
the unit law up to definitional equality. This constraint on \( \cat{C} \) 
automatically makes \( \widehat{\func{F₀}} \) a strict presheaf:
\begin{align*}
    g\ \func{∙}\ (f\ \func{∙}\ \widehat{x}) & = (f\ \func{∘}\ g)\ \func{∙}\ \widehat{x} \\
    (\func{id}\ a)\ \func{∙}\ \widehat{x} & = \widehat{x}.
\end{align*}
Thus, \( \widehat{\func{F₀}} \) can be seen as an object of \( \catsPsh{\cat{C}} \),
and since \func{F₁} satisfies the functoriality equations up to a defininitional equality, 
our embedding \( x \mapsto \widehat{x} \) commutes with restriction:
\( f\ \func{∙}\ \widehat{x} \) is convertible to the embedding
of \( \func{F₁}\ f\ x \). 
% 
This means that we have a \emph{strict} natural tranformation 
\( F \to \widehat{\func{F₀}} \), which effectively replaces the 
functoriality given by \( \func{F₁} \) with a functoriality operation
derived from composition in \( \cat{C} \).

The strict presheaves that can be expressed as \( \widehat{X} \) for some 
\( X : \func{Obj} \to \func{Set} \) 
% 
(that we will call \defnote{cofree}{In the next section, we present 
presheaves as coalgebras, and presheaves of the form \( \widehat{X} \) 
will be cofree coalgebras.} presheaves) 
% 
are remarkably easy to handle, since their functoriality operation is automatically
provided by composition and will satisfy the equations up to a definitional equality.
% 
And as we just explained, every strict presheaf can be embedded into a
cofree presheaf.
% 
This leads us to the definition of a \emph{prefascist type} as a sub-presheaf 
of a cofree presheaf:

\ExecuteMetaData[chapters/literate-agda/Prefascist.tex]{prefascist}

The record that defines a prefascist type contains a type family 
\( \field{F₀} \) on the objects of \( \cat{C} \), and a proof-irrelevant 
predicate on the ``completed'' family \( \widehat{\field{F₀}} \).
% 
This definition is intended to represent the subpresheaf of \( \widehat{\field{F₀}} \)
that is obtained by only considering the elements of \( \widehat{\field{F₀}}\ a \) 
that satisfy the proof-irrelevant predicate \( \field{Fε} \) for all their 
restrictions.

Given a prefascist type \( F \), we can define the type of elements of \( F \)
over an object \( a \) as follows:

\ExecuteMetaData[chapters/literate-agda/Prefascist.tex]{elem}

This record contains an element \field{x₀} of \( \widehat{\field{F₀}}\ a \) and
a proof \field{xε} that all its restrictions satisfy the predicate \field{Fε}.
% 
We can define a functoriality operation on the elements of a prefascist type, by 
using composition on \field{x₀} and \field{xε}. If \( f \) is a morphism from \( b \)
to \( a \), we define
% 
\[
\begin{array}{lcl}
    f\ \func{∙}\ \_ & : & \func{elem}\ F\ a\ \AgdaSymbol{→}\ \func{elem}\ F\ b \\
    f\ \func{∙}\ \func{⟨}\ \field{x₀}\ \func{,}\ \field{xε}\ \func{⟩} & 
        := & \func{⟨}\ f\ \func{∙}\ \field{x₀}\ \func{,}\ f\ \func{∙}\ \field{xε} \ \func{⟩}  
\end{array}
\]

And since composition in \( \cat{C} \) is strictly associative, the prefascist 
types are strict presheaves.
Finally, morphisms of prefascist types are defined as follows:

\ExecuteMetaData[chapters/literate-agda/Prefascist.tex]{morphism}

This definition might look daunting at first, but it simply imitates
the definition of an element of \( G \) with an element of \( F \) in the 
context: ``morphisms are generalized elements''. We invite the reader
to check that morphisms are in fact strict natural transformations (meaning 
that the naturality holds by definition).

Thus, our prefascist types form a remarkable subcategory of \( \catsPsh{\cat{C}} \) that
can be defined in a minor extension of \MLTT.
% 
Naturally, this begs the question of which presheaves can be encoded as
prefascist sets. In the next section, we will see that prefascist types
are equivalent to the full category of presheaves in a set-truncated theory, 
but unfortunately falls short of it when homotopical information 
becomes relevant.

\subsection{Categorical Perspective}

In this section, we revisit prefascist types on \( \cat{C} \) from 
the perspective of category theory. We assume that the reader is familiar 
with the theory of monads and adjunctions.

While the problem of defining the category of presheaves of types on 
\( \cat{C} \) in type theory is rather delicate, there are 
particular cases in which this problem becomes much simpler.
% 
In particular, if the category is \defnote{discrete}{A category is discrete if 
it has only identity morphisms.}, then presheaves of types on \( \cat{C} \) are 
just types indexed by \( \obj{\cat{C}} \) and morphisms are indexed functions.
This category is simple enough to define in intensional type theory, and all of 
the equations are (vacuously) verified up to definitional equality.

Now, remark that every category \( \cat{C} \) contains a discrete subcategory
\( \cat{C}_0 \), which has the same objects as \( \cat{C} \) but only identity
morphisms. The embedding \( \cat{C}_0 \hookrightarrow \cat{C} \) gives rise
to a ``pullback'' functor
\[
\iota^* : \catsPsh{\cat{C}} \to \catsPsh{\cat{C}_0}
\]
which simply forgets the functoriality of a strict presheaf. 
\sideremark{\( \iota^* \) also has a left adjoint, thus both \( \iota^* \) and \( \iota_* \) preserve limits.}
This functor has a right adjoint \( \iota_* \), given by
\[
\begin{array}{lcl}
\iota_* & : & \catsPsh{\cat{C}_0} \to \catsPsh{\cat{C}} \\
\iota_* X (a) & := & \AgdaSymbol{∀}\ (b : \func{Obj})\ (f : \func{Hom}\ b\ a)\ \AgdaSymbol{→}\ X\ b
\end{array}
\]
which is the same as our construction \( \widehat{X} \) from the previous 
section. 

\paragraph*{The presheaf comonad}
By composing both sides of the adjunction, we get a comonad 
\( \iota^* \iota_* \) on \( \catsPsh{\cat{C}_0} \), which we can 
define internally in \MLTT. 
% 
The comultiplication of \( \iota^* \iota_* \) corresponds to composition of
morphisms in \( \cat{C} \):
\[
\begin{array}{lcl}
\nu_X & : & \AgdaSymbol{∀}\ (a : \func{Obj}) \to \iota^* \iota_* X(a) \to \iota^* \iota_* \iota^* \iota_* X(a) \\
\nu_X(a) & := & \lambda\ (x : \iota^* \iota_* X(a))\ (b : \func{Obj})\ (f : \func{Hom}\ b\ a)\\
& & \phantom{\lambda\ (x : \iota^* \iota_* X(a))}\ (c : \func{Obj})\ (g : \func{Hom}\ c\ b)\ .\ x\ c\ (f\ \func{∘}\ g)
\end{array}
\]
And its counit is derived from the identity morphism of \( \cat{C} \):
\[
\begin{array}{lcl}
\varepsilon_X & : & \AgdaSymbol{∀}\ (a : \func{Obj}) \to \iota^* \iota_* X(a) \to X(a) \\
\varepsilon_X(a) & := & \lambda\ (x : \iota^* \iota_* X(a))\ .\ x\ a\ (\func{id}\ a).
\end{array}
\]
Since \( \cat{C} \) is a strict category, it follows that \( \iota^* \iota_* \) 
satisfies the monad equations by definition: the associativity of the 
comultiplication corresponds to associativity of the composition in \( \cat{C} \) 
and the unit laws for the comultiplication correspond to the unit laws for 
\( \cat{C} \).

The comonad \( \iota^* \iota_* \) is particularly interesting to us, because it 
has the category of presheaves as its category of coalgebras.
% 
\begin{theorem}
  The category of strict type-theoretic presheaves \( \catsPsh{\cat{C}} \) is equivalent 
  to the category of strict type-theoretic
  coalgebras of the the comonad \( \iota^* \iota_* \). In other words, \( \iota^* \dashv \iota_* \) is a 
  comonadic adjunction.
\end{theorem}
\begin{proof}
  A strict coalgebra for \( \iota^* \iota_* \) is by definition an collection of types 
  \( F \) indexed by the objects of \( \cat{C} \), along with a 
  collection of functions
  \[ 
     F(a) \to \AgdaSymbol{∀}\ b\ (f : \func{Hom}\ b\ a)\ \to\ F(b)
  \]
  that satisfy the two coalgebra equations by definition, which are respectively the 
  compatibility with composition and identities.
\end{proof}
This decomposition of the theory of presheaves into the adjunction 
\( \iota^* \dashv \iota_* \) is already implicitly present in the work of Jaber 
\etal~\sidecite{Jaber16} where they analyze presheaves under the lens
of call-by-push-value.

\paragraph*{Cofree presheaves}
% 
As we already saw, defining the category of coalgebras of \( \iota^* \iota_* \) 
(\ie the category of presheaves of \( \cat{C} \)) in \MLTT is problematic because of the coalgebra equations.
% 
However there is an important subcategory that we can define in type theory: the
co-Kliesli category of the adjunction, which is equivalent to the category of 
cofree coalgebras.
%
The objects of the co-Kliesli category are presheaves on \( \obj{\cat{C}_0} \), 
and the hom-types are given by
\[
    \hom[\mathrm{coK}]{F}{G} := \AgdaSymbol{∀}\ a\ \to \iota^* \iota_* F(a) \to G(a).
\]
The composition of two morphisms \( \theta \) and \( \xi \) in the co-Kliesli 
category is obtained by computing \( \theta \circ (\iota^* \iota_* \xi) \) in 
\( \catsPsh{\cat{C}_0} \) and composing it with the comultiplication. This 
composition is strictly associative, because all the laws of the comonad 
\( \iota^* \iota_* \) hold by definition.

The subcategory of cofree presheaves is convenient to define and manipulate in 
type theory, but it supports only a small portion of the structure
of the category \( \catPsh{\cat{C}} \). 
% 
In general, the coproduct of two cofree presheaves is not a cofree presheaf.

\paragraph*{Prefascist types}
% 
In the previous section we noticed that any presheaf \( X : \catPsh{\cat{C}} \) 
has a natural transformation that maps it into a cofree presheaf, 
given by the unit of the adjunction \( \iota^* \dashv \iota_* \).
\[
\eta_X : X \to \iota_* \iota^* X
\]
%
This
\sideremark{In the the record \func{prefascist} of the previous section, we did
not ask for \field{Fε} to be preserved by the restriction of \( \widehat{\field{F₀}} \).
We made up for this in the definition of \func{elem} by asking that \field{Fε}
applies to all the restrictions of the elements.}
% 
led us to define a prefascist type as a sub-presheaf of a cofree presheaf,
\ie a pair of a family of types \( F_0 : \catPsh{\cat{C}_0} \) and a propositional
predicate on \( \iota^* \iota_* F_0 \) that is preserved by restriction.

And thus we come back to our question: can every strict presheaf \( X \) be 
presented as a prefascist type?
The following lemma states that it is indeed the case if we are working with
an observational type theory such as the theory \SetoidCC of \cref{ch:observational}.
% 
\begin{theorem}
  Assume that our type theory supports function extensionality and has a 
  proof-irrelevant equality.

  Then the category of strict presheaves \( \catsPsh{\cat{C}} \) is equivalent
  to the category of prefascist types, whose objects are terms of type \func{prefascist}
  and whose morphisms are terms of type \func{pf\_hom} quotiented by propositional 
  equality.
\end{theorem}
\begin{proof}
  We construct an equivalence of categories by hand.
  Given a strict presheaf \( X \), the unit \( \eta_X \) is a monomorphism from 
  \( X \) to a cofree presheaf. Thus we can define a prefascist type as follows: 
  \[
    \begin{array}{lcl}
        \field{F₀} & := & \iota^* X \\
        \field{Fε} & := & \lambda\ (a : \func{Obj})\ (x : \iota^* \iota_* \iota^* X(a))\ .\ x\ \func{≡}\ \eta_X\ (x\ (\func{id}\ a)).
    \end{array}
  \]
  Conversely, given a prefascist type \( Y : \func{prefascist} \), we already 
  know that it naturally has the structure of a strict presheaf, by defining
  \[
    Y(a) := \func{elem}\ Y\ a.
  \]
  Now it only remains to check that this defines an equivalence of categories, 
  which is straightforward since we have function extensionality and UIP.
\end{proof}

Thus, if we work in observational type theory, prefascist types are an 
interesting alternative to the naive definition of presheaves: we get an 
equivalent category, and we gained definitional functoriality laws in the 
process.

Unfortunately, this equivalence will not hold if we work without UIP and 
\sideremark{It is still true that \( \eta_X\ a\ x\ \func{≡} \eta_X\ a\ y \) 
    implies \( x\ \func{≡}\ y \), but this is not sufficient to define a
    monorphism in the absence of UIP.}
interpret types as spaces. In that case, the unit \( \eta_X \) of the 
adjunction \( \iota^* \dashv \iota_* \) is \emph{not} a monomorphism in general, 
and it is not true that every presheaf is a sub-presheaf of a cofree presheaf. 
% 
In the \( \infty \)-groupoid model, the category of prefascist types sits strictly 
inbetween the category of presheaves of hsets and the category of presheaves of types.
% 
We have no clear characterization of which presheaves can be expressed as 
prefascist types.

\subsection{Strictifying Categories}\label{sec:cat-strictification}

Up until now, we worked under the assumption that we have a definition of
a category \( \cat{C} \) that satisfies the associativity of composition
and the unit laws up to definitional equality.
% 
This seems a bit artificial: most categories we can encounter in the wild 
do not have such a strong property, and what good are our prefascist types if
we can only define them on a very limited subset of categories?
% 
However in this section we show that any small category can be presented as
a strict category, given that we have access to a sort \func{Prop} of strict
propositions.

Assume we are given a non-strict category \( \cat{C} \) consisting of a type
of objects \func{Obj}, a \emph{hset} of morphisms for any two objects \( a \) 
and \( b \), and a composition and identities with equations that hold only
up to a propositional equality.

\ExecuteMetaData[chapters/literate-agda/Presheaf.tex]{nonstrictcategory}

We can use the Yoneda lemma to replace the hom-set \( \func{Hom}\ a\ b \) with 
the isomorphic hset \( \hom[\catPsh{\cat{C}}]{\yo(a)}{\yo(b)} \). This way,
morphisms become natural transformations of presheaves, which can be written as
dependent functions with a naturality condition in \func{Prop}. This trick allows
us to define a strict category \( \cat{C}' \) from \( \cat{C} \) as follows:
\[
\begin{array}{lcl}
\func{Obj'} & : & \func{Set} \\
\func{Obj'} & := & \func{Obj} \\
\\
\func{Hom'} & : & \func{Obj'} \to \func{Obj'} \to \func{Set} \\
\func{Hom'}\ a\ b & := & \func{Σ}\ (\theta : \func{∀}\ (c : \func{Obj}) \to \func{Hom}\ c\ a \to \func{Hom}\ c\ b)\ .\\
& & \phantom{\func{Σ}}\ \func{∀}\ (c\ d\ : \func{Obj})\ (f : \func{Hom}\ d\ c)\ (g : \func{Hom}\ c\ a)\\
& & \phantom{\func{Σ}\ \func{∀}}\ \to \func{∥}\  (\theta\ c\ g)\ \func{∘}\ f\ \func{\equiv}\ \theta\ d\ (g\ \func{∘}\ f)\ \func{∥ₛ}\\
\\
\func{id'} & : & (a : \func{Obj'}) \to \func{Hom'}\ a\ a \\
\func{id'}\ a & := & \func{⟨} \lambda\ (c : \func{Obj})\ (f : \func{Hom}\ c\ a)\ .\ f\enskip \func{;}\enskip \lambda\ \_\ \_\ \_\ \_\ .\ \func{refl}\ \func{⟩}\\
\\
\func{\_∘'\_} & : & \{a\ b\ c : \func{Obj'}\} \to \func{Hom'}\ b\ c \to \func{Hom'}\ a\ b \to \func{Hom'}\ a\ c\\
\func{⟨} \theta_f\ \func{;}\ h_f\func{⟩}\ \func{∘'}\ \func{⟨} \theta_g\ \func{;}\ h_g\func{⟩} & := & \func{⟨} \lambda\ (d : \func{Obj})\ (h : \func{Hom}\ d\ a)\ .\ \theta_f\ d\ (\theta_g\ d\ h)\enskip \func{;}\enskip [...] \func{⟩}
\end{array}
\]

With this definition, composition of two morphisms becomes a composition
of functions with a manipulation of proof-irrelevant conditions, which is 
associative by definition. 
Likewise, the unit laws correspond to composition with an identity function,
which is unital by definition.
Therefore our replacement for \( \cat{C} \) is a strict category.

But is \( \cat{C}' \) actually equivalent to \( \cat{C} \)?
It is the case if we interpret types as classical sets, or as 
\( \infty \)-groupoids.
% 
But proving the equivalence in the internal language of type theory is not
possible in general, unless we have enough structure to recover a proof of 
\( a\ \func{\equiv}\ b \) from a proof of \( \func{∥}\ a\ \func{\equiv}\ b\ \func{∥ₛ} \)
---for instance, if we are working in observational type theory.

\subsection{The Prefascist Translation}
\label{sec:PMP_translation}

In the context of a theory with UIP, most of the structure from categories of
presheaves can be replicated for prefascist types in type theory. For instance,
the category of prefascist types has products, pullbacks, exponential object,
coproducts, \etc
% 
In his prefascist paper~\sidecite{pedrot:prefascist}, Pédrot uses prefascist 
types to define a presheaf model of \MLTT as a \emph{syntactic 
translation}~\sidecite{boulier:hal-01445835}.

A prototypical syntactic translation is given by two mappings \( [\_] \) and
\( \llbracket \_ \rrbracket \) from the syntax of \MLTT to the syntax of 
\MLTT such that
% 
\begin{enumerate}
\item if the judgment \( \Gamma \vdash t : A \) is derivable in \MLTT, then 
the judgment
\( {\llbracket \Gamma \rrbracket_{\mathsf{Con}} \vdash [t] : \llbracket A \rrbracket} \)
is derivable in \( \MLTT \), 
\item if \( \Gamma \vdash A \equiv B \) then
\( {\llbracket \Gamma \rrbracket_{\mathsf{Con}} \vdash \llbracket A \rrbracket \equiv \llbracket B \rrbracket} \),
\item if \( \Gamma \vdash t \equiv u : A \) then
\( {\llbracket \Gamma \rrbracket_{\mathsf{Con}} \vdash [t] \equiv [u] : \llbracket A \rrbracket} \), and
\item the translation of the empty type \( \llbracket \bot \rrbracket \) has no 
inhabitants in the empty context.
\end{enumerate}
% 
\sideremark[*-8]{\( \llbracket \_ \rrbracket_{\mathsf{Con}} \) applies
\( \llbracket \_ \rrbracket \) to all the types in a context.}

The prefascist model of Pédrot is slightly more involved: in particular, the target theory
is not \MLTT but \( \sMLTT \), a theory that extends \MLTT with a sort of 
strict propositions and supports a proof-irrelevant equality with large
elimination \emph{à la} \Lean.
% 
With this additional structure in the target theory, the prefascist model
interprets every type as a prefascist set, and every term of type \( A \) as a 
natural transformation from the interpretation of the context to the 
interpretation of \( A \).

In other words, prefascist types make it possible to phrase the traditional
presheaf models in the language of intensional type theory, in a way that 
preserves computation.

% \section{Higher Prefascist Types}